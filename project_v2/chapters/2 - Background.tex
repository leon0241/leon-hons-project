\documentclass[../hons_project.tex]{subfiles}
\begin{document}

\section{Process Algebra}
Concurrency has been studied in many different ways, though with the earliest  the 1960s with some notable models being Petri nets, or the Actor Model. Process Algebras are one such method of modelling a Concurrent System, where the process is modelled in such a way that it is akin to the Universal Algebras of mathematics - in which operations are defined in an axiomatic approach to create a structurally sound way of defining concurrent systems. \citep{baetenBriefHistoryProcess2005} It is easily possible to model simple systems as a flow chart or diagram as you will be able to see throughout this paper, but a formal approach like process algebras will make way for modelling more complex systems, and lays the groundwork to provide a solid foundation to prove and base claims for such systems.

A simple example in action is a process algebra where we only consider the alternative composition operator $+$, where applied to a process $a + b$ means ``Choose $a$, or choose $b$''. Process algebras can typically be modelled in a \textit{Process Graph}, which are diagrams that employ ``states'', and ``actions'' to show the traces, or paths, that a process can take. In this case, the process $a + b$ can be modelled in the following way:

% https://q.uiver.app/#q=WzAsNixbMSwxLCJcXGNpcmMiXSxbMCwyLCJcXGNpcmMiXSxbMiwyLCJcXGNpcmMiXSxbMiw1XSxbMSw1LCJcXGJ1bGxldCJdLFsxLDBdLFswLDEsImEiLDJdLFswLDIsImIiXSxbMyw0XSxbNSwwXV0=
\[\begin{tikzcd}[cramped,column sep=small]
	& {} \\
	& \circ \\
	\circ && \circ \\
	\arrow[from=1-2, to=2-2]
	\arrow["a"', from=2-2, to=3-1]
	\arrow["b", from=2-2, to=3-3]
\end{tikzcd}\]

Where the graph begins at the top into the first node, and then can either progress to the left node via the action $a$, or the right node via the action $b$. $a$ and $b$ are the actions, e.g. ``'' and ``turn right'', while the nodes are the states, e.g. ``apple'' and ``water''

The axioms of the $+$ operator of BPA are as follows:
\begin{itemize}
    \item \textbf{Commutativity}: $a + b$
    \item \textbf{Associativity}: $(a + b) + c = a + (b + c)$
    \item \textbf{Idempotency}: $a + a = a$
\end{itemize}
Comparable to the operation axioms of a Group or Ring in Mathematics, every other operation in a process algebra is constructed similarly. In practice, most process algebras will have some form of alternative composition, but this is a very simplified example and the developed process algebras that exist are designed to handle a lot more complex situations such as unobservable actions, commonly referred to as $\tau$-actions, recursion, which lets a process repeat itself or other processes, and deadlock, which is a state where no desirable outcomes can be reached.

There are many process algebras that exist, the most famous and seminal being $\csp$ \citep{brookesTheoryCommunicatingSequential1984}, CCS \citep{milnerCalculusCommunicatingSystems1980}, and ACP \citep{bergstraProcessAlgebraSynchronous1984}, \citep{bergstraACPtUniversalAxiom1989}, with some other popular calculi being the $\pi$-calculus and its various extensions  \citep{MILNER19921}, \citep{parrowFusionCalculusExpressiveness1998}, \citep{abadiCalculusCryptographicProtocols1999} which have been used to varying degrees in fields like Biology, Business, and Cryptography, or the Ambient Calculus \citep{cardelliMobileAmbients1998} which has been used to model mobile devices.

\section{Encodings of Process Algebra}

With the growing number of process algebras, one might begin to ask if there is a way of comparing different process algebra to each other to find the single best one, as a parallel to Turing Machines and the Church-Turing thesis. However, the wide range of applications that different process algebra are used for makes that rather impractical, and the goal of unifying all process algebra into a single theory seems further and further away as more process algebras for even more specified tasks get created.

A more reasonable approach is to compare different process algebras and their expressiveness, two main relevant methods being \textit{absolute} and \textit{relative} expressiveness.\citep{parrowExpressivenessProcessAlgebras2008} Absolute expressiveness is the idea of comparing a specific process algebra to a question and seeing if it can solve the problem - e.g. if a process algebra is Turing Complete. However, this merely biparts different algebra - the process algebra that are able to solve a specified problem, and the ones who aren't \citep{gorlaUnifiedApproachEncodability2010}. Therefore, the question of relative expressiveness - i.e. how one language compares to another is a lot more useful in terms of categorising different process algebras by expressiveness.

% TODO: soemthing like thte tarnslation of processes can be done

A well studied way of comparing expressiveness is through an ``encoding'', and whether an algebra can be translated from one to another, but not vice versa \citep{petersComparingProcessCalculi2019}. The general notion of an encoding is not defined by clear boundaries, and the criterion for a valid encoding may vary from language to language, but work has been made to try and generalise the notion of a ``valid'' encoding \citep{gorlaUnifiedApproachEncodability2010}, \citep{DBLP:conf/fossacs/Glabbeek18}.

\section{\texorpdfstring{$\csp$}{CSP}}
$\csp$ (Communicating Sequential Processes) \citep{brookesTheoryCommunicatingSequential1984} is a Process Algebra developed by Tony Hoare based on the idea of message passing via communications. It was developed in the 1980s and was one of the first of its kind, alongside CCS by Milner. $\csp$ uses the idea of action prefixing which is where operators are of the syntax $a \to P$, where $a$ is an event and $P$ is a process. 

% A proposed grammar of $\csp$ taken from [some sources] can be expressed as follows
% \begin{align*}
%    P, Q ::= &\mathrm{STOP} \mid \div \mid a\to P \mid P \intchoice Q \mid P \extchoice Q \mid P \sliding Q \mid \\
% 	&P \pcomp Q \mid P \conceal A \mid f(P) \mid P \interrupt Q \mid P \interrupt Q \mid
% \end{align*}
% where the operators are: \textit{inaction}, \textit{divergence}, \textit{action prefixing}, \textit{internal choice}, \textit{external choice}, \textit{sliding choice}, \textit{parallel composition}, \textit{concealment}, \textit{renaming}, \textit{interrupt}, and \textit{throw}. Compared to [source], this leaves out [sources]


\section{ACP}
ACP (Algebra of Communicating Processes) is a Process Algebra developed by Jan Bergstra and Jan Willem Klop \citep{bergstraProcessAlgebraSynchronous1984}. Compared to $\csp$, ACP is built up with an axiomatic approach in mind which does away with the idea of action prefixing and instead can allow for unguarded operations.

The biggest difference between $\csp$ and $\acp$ is the difference in Communication of parallel processes, as $\csp$ uses conjugate actions to form communications, while $\acp$ uses a separately defined communication function that can be defined over any action. $\acpt$ \citep{bergstraACPtUniversalAxiom1989} is an extension of ACP that includes an extra action $\tau$ which is used to represent actions that are unobservable, or changeable, from a human perspective.

% The grammar of ACP$_{\tau}$ as taken from \citep{bergstraACPtUniversalAxiom1989} is defined as such:
% \begin{align*}
%    P, Q ::= a \mid \delta \mid E + F \mid E . F \mid E \merge F \mid E \leftmerge F \mid E | F \mid \restrict(E) \mid \abtau(E)
% \end{align*}
% where the operators are: \textit{action}, \textit{deadlock}, \textit{alternative composition}, \textit{sequential composition}, \textit{merge}, \textit{left merge}, \textit{communication merge}, \textit{encapsulation}, \textit{abstraction}


{\huge [WIP]}

\end{document}
