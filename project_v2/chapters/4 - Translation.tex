\documentclass[../hons_project.tex]{subfiles}
\begin{document}

% ╭──────────────────────────────────────────────────────────╮
% │              Set a counter for Translations              │
% ╰──────────────────────────────────────────────────────────╯

\newcounter{translations}
\setcounter{translations}{0}

\newcounter{funcs}
\setcounter{funcs}{0}

\newcounter{comms}
\setcounter{comms}{0}

\newcounter{opers}
\setcounter{opers}{0}


\section{Direct Translations}\label{sc:direct-translations}
Some of the basic operations of $\csp$ have an equivalent counterpart in ACP, with the only difference being the syntax. These can be easily translated in the following table.
\begin{align}
	\label{trans:stop}\tag{T\thetranslations}\stepcounter{translations}
	\trans{STOP}           & = \delta              \\
	\label{trans:apr}\tag{T\thetranslations}\stepcounter{translations}
	\trans{a \to P}        & = a.\trans{P}         \\
	\label{trans:abstraction}\tag{T\thetranslations}\stepcounter{translations}
	\trans{P \backslash A} & = \tau_{A}{\trans{P}}
	% \trans{\mu p.P} &= \langle X \mid X = \tau.X \rangle
\end{align}

\section{Trivial Translations}\label{sc:trivial-translations}
\begin{itemize}[leftmargin=*]
	\item \textbf{Divergence} is the process that diverges via infinite internal actions, implying that a user can never make a decision past that point. It is defined by the rule $\mathrm{div} \prightarrow{\tau} \mathrm{div}$, and can be directly translated using recursion in ACP in the following rule:
	      \begin{equation}\label{trans:div} \tag{T\thetranslations}\stepcounter{translations}
		      \trans{\mathrm{div}} = \langle X \mid X = \tau.X \rangle
	      \end{equation}
	\item \textbf{Renaming} is an operation that renames actions in processes according to a function. There is no equivalent operator in plain $\mathrm{ACP}_{\tau}$, with the closest operation being $\tau_{I}(P)$ which abstracts actions in $I$ to internal actions. This is possible in our extension $\acptf$, and in fact our translation is trivially
	      \begin{equation}\label{trans:renaming}
		      \tag{T\thetranslations}\stepcounter{translations}
		      \trans{f(P)} = f(\trans{P})
	      \end{equation}

	\item \textbf{Internal Choice} is an operation that emulates a choice of actions that isn't decided by the user, for example the result of flipping a virtual coin.
	      $\csp$ differentiates from external choice and internal choice, while in ACP the alternative choice operator $+$ handles both choice, with internal actions satisfying the $+$ operator. With the internal action $\tau$, a translation for $\csp$ Internal choice into ACP is easily written as
	      \begin{equation}\label{trans:intchoice}
		      \tag{T\thetranslations}\stepcounter{translations}
		      \trans{P \sqcap Q} = \tau.\trans{P} + \tau.\trans{Q}
	      \end{equation}
\end{itemize}

% TODO: Write about the terminology ``satisfying''

The above translations are all valid up to Strong Bisimilarity.

\section{Helper Operators for the language \texorpdfstring{$\acptf$}{ACPtf}}

\subsection{Subsets of \texorpdfstring{$\Sigma$}{Sigma}}
Working in $\acptf$ as defined in Definition \ref{sc:acp}, we first recall that $\acptf$ is paramaterised by a set of actions $\Sigma$. Also recall that $\Sigma_{\tau}$ is defined as the set $\Sigma \cup \{\tau\}$. We start by defining some subsets of $\Sigma$ which we will use in our encodings.

\begin{dfn}[Subsets of A]{dfn:sets}{}
	The set $\Sigma\in \mathbb{T}_{\acptf}$ is the set of all possible actions.
	\begin{itemize}
		\item $\Sigma_{0} \subseteq \Sigma$ is the set of actions that get used in processes
		\item $A \subseteq \Sigma_{0}$ is a set of target actions. This is used in operators such as $\csp$ Parallel Composition, which only communicates over a set.
		\item $H_{0} = \Sigma - \Sigma_{0}$ is the set of working space operators, or any other action that doesn't get used in processes.
		\item $\mathscr{H} \subseteq H_{0}$ is a selectively chosen set from $H_{0}$ to aid a translation.
		\item $H_{1} = \Sigma_{0} \uplus \mathscr{H}$ is the set of actions, plus any actions of $\mathscr{H}$. This can also be thought of the set of actions used in a translation.
	\end{itemize}

	In general, $A \subseteq \Sigma_{0} \subseteq H_{1} \subseteq \Sigma$, and $\mathscr{H} \subseteq H_{0} \subseteq \Sigma$.

	We also have $\Sigma_{0} \cap H_{0} = \emptyset$, and by extension, $\Sigma_{0} \cap \mathscr{H} = \emptyset$
\end{dfn}


\begin{figure}[!ht]
	\centering
	\import{diagrams/}{triggering.tex}
\end{figure}
\subsection{Triggering}\label{ssec:triggering}
We define an operator $\Gamma(P)$ that emulates the Triggering operator of MEIJE \cite{austryAlgebreProcessusSynchronisation1984, desimoneHigherlevelSynchronisingDevices1985}. For a trace $a.b.\cdots$ on a process $P$, the triggering operator can be represented as an operator that tags the first action of a process.
First, we define a function $f_{\mathtt{trig}}$ and communications for two actions $\mathtt{first}$ and $\mathtt{next}$ that are from $H_{0}$.

\begin{dfn-s}[Communications]{dfn:comm-triggering}{}
Define communications where:
\begin{equation}\label{comm:trig}\tag{F\thecomms}\stepcounter{comms}
	a | \mathtt{first} = a_{\mathtt{first}} \quad
	a | \mathtt{next} = a_{\mathtt{next}}
\end{equation}
\end{dfn-s}

\begin{dfn-s}[Triggering Function]{dfn:f1}{}
Define $f_{\mathtt{trig}}: \Sigma_{\tau} \to \Sigma_{\tau}$ where:
\begin{equation}\label{func:trig}\tag{F\thefuncs}\stepcounter{funcs}
	f_{\mathtt{trig}}(\alpha) = \begin{cases}
		a_{\mathtt{ini}} & \text{if } \alpha = a_{\mathtt{first}} \\
		a                & \text{if } \alpha = a_{\mathtt{next}}  \\
		\alpha           & \text{otherwise}
	\end{cases}
\end{equation}
\end{dfn-s}

We use the notation of $a^{\infty}$ as syntactic sugar to mean $\langle X \mid X = a.X \rangle$. Using the sets defined in Definition \ref{dfn:sets}, we can now define $\Gamma(P)$ as such:

\begin{dfn}[Triggering in ACP]{dfn:acp-triggering}{}
	The \textbf{Triggering} operator is defined as the following process equation:
	\begin{equation}\label{oper:triggering}\tag{O\theopers}\stepcounter{opers}
		\Gamma(P) := f_{\mathtt{trig}}\bigl[\partial_{H_{1}}(P \merge \mathtt{first}.(\mathtt{next}^{\infty}))\bigr]
	\end{equation}
	$\Gamma(P)$ turns each trace $a.b.c.\dots$ of a process $P$ into the trace
	\[a_{\mathtt{ini}}. b. c. \dots\]
\end{dfn}

This works in the following method:
\vspace{-5pt}
\begin{enumerate}[label=\alph*)]
	\item Merge the process $P$ with the process $\mathtt{first}.\mathtt{next}.\mathtt{next}\dots$. Via Definition \ref{dfn:comm-triggering}, this will produce a lattice of $P$ and $\mathtt{first}.(\mathtt{next}^{\infty})$, with communications on every square, but most importantly, a chain of communications going through the center of the form
	    \begin{equation}\label{eq:fnn}
			a_{\mathtt{first}}. b_{\mathtt{next}} . c_{\mathtt{next}} \dots
	    \end{equation}
	\item Apply the Restriction operator $\restrict$ to the actions in $H_{1}$, as defined in Definition \ref{dfn:sets}. Since all the actions in both the processes $P$ and $\mathtt{first}.(\mathtt{next}^{\infty})$ are in the set $H_{1}$, this effectively restricts both sides of the left merge, leaving only communications from the initial state. This leaves \cref{eq:fnn} as the only remaining trace.
	\item Apply $f_{\mathtt{trig}}$ as defined in \cref{func:trig} to \cref{eq:fnn}. , the final result is
	    \begin{equation}\label{eq:gamma-result}
			a_{\mathtt{ini}}. b . c \dots
		\end{equation}

	    The process is now exactly as stated in Definition \ref{dfn:acp-triggering}.
\end{enumerate}

Note that since $\tau\not\in \Sigma$, the restriction operator $\partial_{H_{1}}$ will not affect $\tau$. Additionally, since $\tau$ does not communicate with any actions, Step b effectively meaning ``some amount of $\tau$ steps followed by the diagonal trace immediately following''. This results in traces such as the one shown below:
\[\Gamma(\tau.b.c) = \tau.b_{\mathtt{ini}}.c\]
effectively skipping $\tau$'s, then acting the same as processes that don't start with a $\tau$.

\begin{figure}[!h]
   \centering
	\import{diagrams/}{triggering-tau.tex}
   \caption{Example of Triggering operator with internal actions}
   \label{fig:triggering-tau}
\end{figure}

\subsection{Associativity and Postfix Function}\label{ssec:postfix}
Shown in \cref{table:acpt-communication} is a list taken from the axioms of $\acpt$ \cite{bergstraACPtUniversalAxiom1989} of the rules associated with the communication operator $\mid$. Bearing these axioms in mind, we have the potential to run into problems with our communications, especially with the associativity rule 
\[(a \mid b) \mid c = a \mid (b \mid c)\]

\begin{table}[ht!]
	\centering
	\begin{tabular}{ |c| }
		\hline
		\textbf{Communication function in $\acpt$} \\
		\hline
		$a | b = b | a$                            \\
		$(a | b) | c = a | (b | c)$                \\
		$\delta | a = \delta$                      \\
		\hline
	\end{tabular}
	\caption{Axioms of $\acpt$ Communication}
	\label{table:acpt-communication}
\end{table}

\vspace{-10pt}
For example, in our proposed translation for the $\csp$ External Choice operator, as shown in \cref{ssec:external-choice}, a simplified version of the translation could have the following communications:
\[a | \mathtt{first} = a_{\mathtt{first}} \qquad a | \mathtt{next} = a_{\mathtt{next}} \qquad a_{\mathtt{first}} | \mathtt{choose} = a\]
The Associativity axiom does not hold true, such as in the following counterexample:
\begin{align*}
	a | \mathtt{first} | \mathtt{choose} & = {\color{blue}(a | \mathtt{first})} | \mathtt{choose} = a_{\mathtt{first}} | \mathtt{choose} = a \\
	 a | \mathtt{first} | \mathtt{choose}                                    & = a | {\color{blue}(\mathtt{first} | \mathtt{choose})} = a | \delta = \delta
\end{align*}
From this example, it is clear that the proposed communications would not satisfy the axioms of $\acpt$. This is also the reason that in Definition \ref{dfn:f1} we have the rule
$f_{\mathtt{trig}}(a_{\mathtt{first}}) = a_{\mathtt{ini}}$ instead of $f_{\mathtt{trig}}(a_{\mathtt{first}}) = a_{\mathtt{first}}$
A preferred list of communications for External Choice is as follows:
\[a | \mathtt{first} = a_{\mathtt{first}} \qquad a | \mathtt{next} = a_{\mathtt{next}} \qquad a_{\mathtt{ini}} | \mathtt{choose} = a\]
From these communications, associativity of the equation now holds:
\begin{align*}
	a | \mathtt{first} | \mathtt{choose} & = {\color{blue}(a | \mathtt{first})} | \mathtt{choose} = a_{\mathtt{first}} | \mathtt{choose} = \delta \\
	a | \mathtt{first} | \mathtt{choose} & = a | {\color{blue}(\mathtt{first} | \mathtt{choose})} = a | \delta = \delta
\end{align*}
It is important to note that $\Gamma(P)$ takes precedence and hence the communication $\mathtt{first} | \mathtt{choose}$ would never occur for our application of External Choice, but the communication function must work over every action regardless of whether it will get used in practice for it to be correct for the axioms of $\acp$. To fix this, we define a compatibility function, $\fdef{post}$. We tag any potential renamings with $a_{\mathtt{post}}$ to act as a filler for communications. Now, our final step will be to rename $a_{post}$ back to $a$ for any affected actions. This works in the following way:
\[a \xrightarrow{\text{Rename for Communication}} a_{\mathtt{tag}} \xrightarrow{\text{Communicate with an action}} a_{\mathtt{post}} \xrightarrow{\text{Rename for final result}} a\]

\begin{dfn}[Postfix function]{dfn:postfix}{}
	Let $f_{\mathrm{post}} : \Sigma_{\tau} \to \Sigma_{\tau}$ where
	\vspace{-5pt}
	\begin{equation}\label{func:postfix}
	\tag{F\thefuncs}\stepcounter{funcs}
	    f_{\mathtt{post}}(\alpha) = \begin{cases}
			a      & \text{ if } \alpha = a_{\mathrm{post}} \\
			\alpha & \text{otherwise}
		\end{cases}
	\end{equation}
\end{dfn}

\section{Translations for the remaining \texorpdfstring{$\csp$}{CSP} Operators}

\subsection{Communications and Functional Renaming}

We define functions for our translation in addition to the ones previously defined in \cref{ssec:triggering,,ssec:postfix}. Also note that these are defined over $\Sigma_{\tau}$ for bookkeeping purposes as internal actions cannot get renamed and therefore stay as a $\tau$ no matter the function.

\begin{dfn}[Helper Functions]{dfn:helper-functions}{}
	Recall the functions $\fdef{trig}$ and $\fdef{postfix}$ defined in \cref{func:trig,,func:postfix}:

	\[f_{\mathtt{trig}}(\alpha) = \begin{cases}
			a_{\mathtt{ini}} & \text{if } \alpha = a_{\mathtt{first}} \\
			a                & \text{if } \alpha = a_{\mathtt{next}}  \\
			\alpha           & \text{otherwise}
		\end{cases} \qquad f_{\mathtt{post}}(\alpha) = \begin{cases}
			a      & \text{ if } \alpha = a_{\mathrm{post}} \\
			\alpha & \text{otherwise}
		\end{cases}\]
	We define functions for the remaining operators below. We use the notation $A_{T}$ to signify a target set, as used in \cref{ssec:parallel-composition,,ssec:throw}
	\begin{itemize}
		\item $\fdef{syn} : \Sigma_{\tau} \to \Sigma_{\tau}$ is a function that renames any actions in the target set $A$. This is used in the translation of Parallel Composition (\ref{ssec:parallel-composition})
			\begin{equation}\label{func:syn}
			\tag{F\thefuncs}\stepcounter{funcs}
			    f_{\mathrm{syn}}(\alpha) = \begin{cases}
				      \alpha_{\mathtt{syn}} & \text{ if } \alpha\in A \\
				      \alpha                & \text{otherwise}
			      \end{cases}
			\end{equation}
		\item $\fdef{origin} : \Sigma_{\tau} \to \Sigma_{\tau}$ is a function that renames actions in a process for use in operators. This is used in the translation of the Interrupt operator (\ref{ssec:interrupt})
			\begin{equation}\label{func:origin}
			\tag{F\thefuncs}\stepcounter{funcs}
			   \fdef{origin}(\alpha) = \begin{cases}
				      a_{\mathtt{origin}} & \text{if } a\in\Sigma \\
				      \tau                & \text{otherwise}
			      \end{cases} 
			\end{equation}
		\item $\fdef{split} : \Sigma_{\tau} \to \Sigma_{\tau}$ is a function that renames actions in a process for use in operators, and also renames actions in the target set $A$. This is used in the translation of the Throw operator (\ref{ssec:throw})
			\begin{equation}\label{func:split}
			\tag{F\thefuncs}\stepcounter{funcs}
			    f_{\mathtt{split}}(\alpha) = \begin{cases}
					a_{\mathtt{split}}  & \text{ if } a\in A                    \\
				    a_{\mathtt{origin}} & \text{ if } a\not\in A,\, a\in \Sigma \\
				    \tau                & \mathrm{otherwise}
			    \end{cases}
			\end{equation}
	\end{itemize}
\end{dfn}

To aid these functions, we also define communications to use in our translation in addition to the ones previously defined in \cref{ssec:triggering}.
\newpage

\begin{dfn}[Communications]{dfn:communications}{}
	Recall the communications for the Triggering operator defined in \cref{comm:trig}:
	\[a | \mathtt{first} = a_{\mathtt{first}} \qquad a | \mathtt{next} = a_{\mathtt{next}} \]
	We define our additional communications. These all communicate to $a_{\mathtt{post}}$ as shown in \cref{ssec:postfix}.
	\begin{itemize}
		\item Communication for the $a_{\mathtt{syn}}$ tag. This is used in the translation of Parallel Composition (\ref{ssec:parallel-composition}).
			\begin{equation}\label{comm:syn}
			\tag{C\thecomms}\stepcounter{comms}
			a_{\mathtt{syn}} | a_{\mathtt{syn}} = a_{\mathtt{post}}
			\end{equation}
		\item Communication for the $a_{\mathtt{ini}}$ tag. This is used in the translation of External Choice (\ref{ssec:external-choice})
			\begin{equation}\label{comm:ini-choose}
			\tag{C\thecomms}\stepcounter{comms}
				a_{\mathtt{ini}} | \mathtt{choose} = a_{\mathtt{post}} 
			\end{equation}

		\item Communication for the $a_{\mathtt{origin}}$ tag. This is used in the translation of the Interrupt and Throw operator (\ref{ssec:interrupt}, \ref{ssec:throw})
			\begin{equation}\label{comm:origin}
			\tag{C\thecomms}\stepcounter{comms}
				a_{\mathtt{origin}} | \mathtt{origin} = a_{\mathtt{post}}    
			\end{equation}

		\item Communications for the $a_{\mathtt{split}}$ tag. \cref{comm:ini-split} is used in the translation of the Interrupt operator (\ref{ssec:interrupt}), and \cref{comm:split-split} is used the translation of the Throw operator (\ref{ssec:throw})
			\begin{align}
				a_{\mathtt{ini}} | \mathtt{split} = a_{\mathtt{post}}\label{comm:ini-split}\tag{C\thecomms}\stepcounter{comms} \\
a_{\mathtt{split}} | \mathtt{split} = a_{\mathtt{post}}\label{comm:split-split}\tag{C\thecomms}\stepcounter{comms}
			\end{align}
	\end{itemize}
\end{dfn}

\subsection{Parallel Composition}\label{ssec:parallel-composition}
The parallel composition $\pcomp$ is defined with the following rules from \cref{table:CSP}:
\begin{equation*}
	\prftree{P \prightarrow{\alpha} P' \quad \scriptstyle(\alpha\not\in A)}{P \pcomp Q \prightarrow{\alpha} P'\pcomp Q} \qquad \prftree{Q \prightarrow{\alpha} Q' \quad \scriptstyle(\alpha\not\in A) }{P \pcomp Q \prightarrow{a} P\pcomp Q'}\qquad \prftree{P \prightarrow{a} P' \, Q \prightarrow{a} Q'\scriptstyle(\alpha\in A)}{P \pcomp Q \prightarrow{a} P' \pcomp Q'}
\end{equation*}
This operator functions mostly the same as the $\acptf$ equivalent of parallel composition, as explained in \cref{ssec:acp-communication}. The one difference is that in $\csp$, the action must be the same in $P$ and $Q$ to communicate, whereas in $\acptf$ communications are defined with a function, $\mid$.

For our encoding, we take $\mathscr{H} = \{\}$, and therefore $H_{1} = \Sigma_{0}$. The goal is to tag actions in the target set $A$, and then define a communication function between identically marked actions. We can do this using the following functions and communications:

\begin{dfn}[Functions and Communications - Parallel Composition]{dfn:comms-pcomp}{}
	As defined in Definitions \cref{dfn:helper-functions,dfn:communications}, we use \cref{func:postfix,,func:syn,,comm:syn}.
	\[f_{\mathtt{syn}}(\alpha) = \begin{cases}
			a_{\mathtt{syn}} & \text{ if } \alpha\in A \\
			\alpha           & \text{otherwise}
		\end{cases} \qquad f_{\mathtt{post}}(\alpha) = \begin{cases}
			a      & \text{ if } \alpha = a_{\mathrm{post}} \\
			\alpha & \text{otherwise}
		\end{cases}\]
	\[a_{\mathtt{syn}} | a_{\mathtt{syn}} = a_{\mathtt{post}}\]
\end{dfn}


We now define our encoding of $\csp$ Parallel Composition $\pcomp$ as the following equation:
\begin{equation}\label{trans:pcomp}
\tag{T\thetranslations}\stepcounter{translations}
	\trans{P \pcomp Q} = \partial_{H_{0}}\Bigl(\fdef{post}\Bigl[\fdef{syn}(\trans{P}) \merge \fdef{syn}(\trans{Q})\Bigr]\Bigr)
\end{equation}

\subsection{External choice}\label{ssec:external-choice}
The external choice operator $\square$ is defined with the following rules from \cref{table:CSP}:
\begin{equation*}\label{eq:csp-definition}
	\prftree{P \prightarrow{a} P'}{P \square Q \prightarrow{a} P'} \qquad \prftree{Q \prightarrow{a} Q'}{P \square Q \prightarrow{a} Q'} \qquad \prftree{P \prightarrow{\tau} P'}{P \square Q \prightarrow{\tau} P' \square Q} \qquad \prftree{Q \prightarrow{\tau} Q'}{P \square Q \prightarrow{a} P \square Q'}
\end{equation*}
In other words, we can take an external choice by the user, and additionally an internal action will still let an external choice be made after the internal move has been made. This differs from the $\acptf$ Alternative Choice operator ($+$), as $+$ will not let you select externally if an internal action is made.

For our encoding, we take $\mathscr{H} = \{\mathtt{first}, \mathtt{next}, \mathtt{choose}\}$, and therefore we have that $H_{1} = A_{0} \uplus \{\mathtt{first}, \mathtt{next}, \mathtt{choose}\}$ as defined in Definition \ref{dfn:sets}.

\begin{dfn}[Functions and Communications - External Choice]{dfn:comms-external}{}
	As defined in Definitions \cref{dfn:helper-functions,dfn:communications}, we use \cref{func:trig,,func:postfix,,comm:ini-choose}.
	\[f_{\mathtt{trig}}(\alpha) = \begin{cases}
			a_{\mathtt{ini}} & \text{if } \alpha = a_{\mathtt{first}} \\
			a                & \text{if } \alpha = a_{\mathtt{next}}  \\
			\alpha           & \text{otherwise}
		\end{cases} \qquad f_{\mathtt{post}}(\alpha) = \begin{cases}
			a      & \text{ if } \alpha = a_{\mathrm{post}} \\
			\alpha & \text{otherwise}
		\end{cases}
	\]
	\[ a_{\mathtt{ini}} | \mathtt{choose} = a_{\mathtt{post}} \]
	Additionally, recall the Triggering operator as defined in \ref{oper:triggering}:
	\[\Gamma(P) := f_{\mathtt{trig}}[\partial_{H_{1}}(P \merge \mathtt{first}(\mathtt{next}^{\infty}))])\]
\end{dfn}


We now define our encoding of $\csp$ External Choice $\square$ as the following equation:
\begin{equation}\label{trans:extchoice}
\tag{T\thetranslations}\stepcounter{translations}
    \trans{P \extchoice Q} = \partial_{H_{0}}\Bigl(\fdef{post}\Bigl[\Gamma[\trans{P}]\, \merge \,\mathtt{choose}\, \merge \,\Gamma[\trans{Q}]\Bigl]\Bigr)
\end{equation}

\subsection{Sliding Choice}\label{ssec:sliding-choice}
The Sliding Choice operator $\triangleright$ is defined with the following rules from \cref{table:CSP}:

\[\prftree{P \prightarrow{a} P'}{P \triangleright Q \prightarrow{a} P'} \qquad \prftree{P \prightarrow{\tau} P'}{P \triangleright Q \prightarrow{\tau} P' \triangleright Q} \qquad P \triangleright Q \prightarrow{\tau} Q\]

In other words, this operator lets you take an external action on $P$. However, at any point before the external action, $P$ may ``time out'' and internally move to $Q$ instead.

For our encoding, we take $\mathscr{H} = \{\mathtt{first}, \mathtt{next}, \mathtt{shift}\}$, and therefore we have that $H_{1} = \Sigma_{0} \uplus \{\mathtt{first}, \mathtt{next}, \mathtt{shift}\}$ as defined in Definition \ref{dfn:sets}.

\begin{dfn}[Functions and Communications - Sliding Choice]{dfn:comms-sliding}{}
	Similarly to External Choice, we use \cref{func:trig,,func:postfix,,comm:ini-choose} As defined in Definitions \cref{dfn:helper-functions,dfn:communications}.
	\begin{align*}
		f_{\mathtt{trig}}(\alpha) & = \begin{cases}
			                              a_{\mathtt{ini}} & \text{if } \alpha = a_{\mathtt{first}} \\
			                              a                & \text{if } \alpha = a_{\mathtt{next}}  \\
			                              \alpha           & \text{otherwise}
		                              \end{cases} \quad
		                          & f_{\mathtt{post}}(\alpha)             = \begin{cases}
			                                                                    a      & \text{ if } \alpha = a_{\mathtt{post}} \\
			                                                                    \alpha & \text{otherwise}
		                                                                    \end{cases}
	\end{align*}
	\[a_{\mathtt{ini}} | \mathtt{choose} = a_{\mathtt{post}}\]
	Additionally, recall the Triggering operator as defined in \ref{oper:triggering}:
	\[\Gamma(P) := f_{\mathtt{trig}}[\partial_{H_{1}}(P \merge \mathtt{first}(\mathtt{next}^{\infty}))])\]
\end{dfn}

We now define our encoding of $\csp$ Sliding Choice $\sliding$ as the following equation:
\begin{equation}\label{func:sliding}
\tag{F\thefuncs}\stepcounter{funcs}
    \trans{P \sliding Q} = \partial_{H_{0}}\Bigl(\tau_{\{\mathtt{shift}\}}\Bigl(f_{\mathtt{post}}\Bigl[ \Gamma(\trans{P}) \merge \mathrm{choose} \merge \mathtt{shift}_{\mathtt{ini}} . \trans{Q}) \Bigr]\Bigr)\Bigr)
\end{equation}

\subsection{Interrupt}\label{ssec:interrupt}

The Interrupt operator $\interrupt$ is defined with the following rules from \cref{table:CSP}:
\[\prftree{P \prightarrow{\alpha} P'}{P \interrupt Q \prightarrow{\alpha} P' \interrupt Q} \qquad \prftree{Q \prightarrow{\tau} Q'}{P \interrupt Q \prightarrow{\tau} P \interrupt Q'} \qquad \prftree{Q \prightarrow{a} Q'}{P \interrupt Q \prightarrow{a} Q'}\]

In other words, we can take an external action from $P$ without satisfying\footnote{Here, we use ``satisfies'' to mean that the operator carries on after an action is executed} the operator, in addition to internal actions from $Q$. However, the moment an external action is made from $Q$, the process can then never return to $P$.

For our encoding, we take $\mathscr{H} = \{\mathtt{first}, \mathtt{next}, \mathtt{origin}, \mathtt{split}\}$, and therefore we have that $H_{1} = \Sigma_{0} \uplus \{\mathtt{first}, \mathtt{next}, \mathtt{origin}, \mathtt{split}\}$ as defined in Definition \ref{dfn:sets}.

\begin{dfn}[Functions and Communications - Interrupt]{dfn:comms-interrupt}{}
	As defined in Definitions \cref{dfn:helper-functions,dfn:communications}, we use \cref{func:trig,,func:postfix,,func:origin,,comm:origin,,comm:ini-split}.
	\begin{align*}
		f_{\mathtt{trig}}(\alpha) & = \begin{cases}
			                              a_{\mathtt{ini}} & \text{if } \alpha = a_{\mathtt{first}} \\
			                              a                & \text{if } \alpha = a_{\mathtt{next}}  \\
			                              \alpha           & \text{otherwise}
		                              \end{cases} \quad
		                          & f_{\mathtt{post}}(\alpha) = \begin{cases}
			                                                                    a      & \text{ if } \alpha = a_{\mathrm{post}} \\
			                                                                    \alpha & \text{otherwise}
		                                                                    \end{cases}\\
		\fdef{origin}(\alpha)     & = \begin{cases}
			                              a_{\mathtt{origin}} & \text{if } a\in\Sigma \\
			                              \tau                & \text{otherwise}
		                              \end{cases}  & \begin{split}
									  a_{\mathtt{origin}} | \mathtt{origin}  = a_{\mathtt{post}}&\\
									  a_{\mathtt{ini}} | \mathtt{split} = a_{\mathtt{post}}&
		                            \end{split}
	\end{align*}
	Additionally, recall the Triggering operator as defined in \ref{oper:triggering}:
	\[\Gamma(P) := f_{\mathtt{trig}}[\partial_{H_{1}}(P \merge \mathtt{first}(\mathtt{next}^{\infty}))])\]
\end{dfn}

We can now define our encoding of the $\csp$ Interrupt operator $\interrupt$ in $\acptf$. We start off with a new helper process, which we will call $\Pi$. This is defined as the recursive equation
\[\Pi = \langle X \mid X = \mathtt{origin}.X + \mathtt{split} \rangle\]
Or, visualised as a process graph:
% https://q.uiver.app/#q=WzAsMyxbMSwwLCJcXGNpcmMiXSxbMywwLCJcXGJ1bGxldCJdLFswLDBdLFswLDEsIlxcbWF0aHR0e3NwbGl0fSJdLFswLDAsIlxcbWF0aHR0e29yaWdpbn0iXSxbMiwwLCIiLDAseyJzaG9ydGVuIjp7InNvdXJjZSI6NTB9fV1d
\[\]

\begin{equation}\label{oper:pi}
\tag{O\theopers}\stepcounter{opers}
    \begin{tikzcd}[column sep=scriptsize, row sep=scriptsize]
		{} & \circ && \bullet
		\arrow[shorten <=8pt, from=1-1, to=1-2]
		\arrow["{\mathtt{origin}}", from=1-2, to=1-2, loop, in=55, out=125, distance=10mm]
		\arrow["{\mathtt{split}}", from=1-2, to=1-4]
	\end{tikzcd}
\end{equation}

We now define our encoding of $\csp$ Interrupt $\interrupt$ as the following equation:
\begin{equation}\label{trans:interrupt}
\tag{T\thetranslations}\stepcounter{translations}
    \trans{P \interrupt Q} = \partial_{H_{0}}\Bigl(\fdef{post}\Bigl[ \fdef{origin}(\trans{P}) \merge \Pi \merge \Gamma(\trans{Q}) \Bigr]\Bigr)
\end{equation}

\subsection{Throw}\label{ssec:throw}

The Throw operator $\throw$ is defined with the following rules:

\[\prftree{P \prightarrow{\alpha} P' \quad \scriptstyle(a\not\in A)}{P \throw Q \prightarrow{\alpha} P' \throw Q} \qquad \prftree{P \prightarrow{a} P' \quad \scriptstyle(a\in A)}{P \throw Q \prightarrow{a} Q }\]

In other words, we can take as many actions in $P$ as we want as long as they aren't contained in a set of actions, $A$. However, the moment an action in $A$ is made, the process then diverts to $Q$. This can be also thought as an error catching operator, with a set of ``error'' actions that switches to an ``exception'' process.

Similarly to Interrupt, for our encoding we take $\mathscr{H} = \{\mathtt{first}, \mathtt{next}, \mathtt{origin}, \mathtt{split}\}$. Therefore, we have $H_{1} = \Sigma_{0} \uplus \{\mathtt{first}, \mathtt{next}, \mathtt{origin}, \mathtt{split}\}$ as defined in \ref{dfn:sets}.

\begin{dfn}[Functions and Communications - Throw]{dfn:comms-throw}{}
	As defined in Definitions \cref{dfn:helper-functions,dfn:communications}, we use \cref{func:postfix,,func:split,,comm:origin,,comm:split-split}.
	\[f_{\mathtt{post}}(\alpha) = \begin{cases}
			a      & \text{ if } \alpha = a_{\mathrm{post}} \\
			\alpha & \text{otherwise}
		\end{cases} \qquad f_{\mathtt{split}}(\alpha) = \begin{cases}
			a_{\mathtt{split}}  & \text{ if } a\in A                    \\
			a_{\mathtt{origin}} & \text{ if } a\not\in A,\, a\in \Sigma \\
			\tau                & \mathrm{otherwise}
		\end{cases}\]
	\[a_{\mathtt{origin}} | \mathtt{origin} = a_{\mathtt{post}} \qquad a_{\mathtt{split}} | \mathtt{split} = a_{\mathtt{post}}\]
\end{dfn}

We can now define our encoding of the $\csp$ Throw operator $\throw$. We employ the use of the same helper process $\Pi$, defined in the translation of the Interrupt operator \ref{oper:pi}.
\[\Pi = \langle X \mid X = \mathtt{origin}.X + \mathtt{split} \rangle\]

From this, our encoding can be written as the following equation:
\begin{equation}\label{trans:throw}
\tag{T\thetranslations}\stepcounter{translations}
    \trans{P \throw Q} = \partial_{H_{0}}\Bigl(f_{\mathtt{post}}\Bigl[ f_{\mathtt{split}}(\trans{P}) | | \Pi.\trans{Q} \Bigr]\Bigr)
\end{equation}





\subsection{Final Translation}

Recall the grammar of the languages $\csp$ as shown in Definitions \cref{dfn:csp}:
\begin{align*}
	P, Q ::= & \mathrm{STOP} \mid \mathrm{div} \mid a\to P \mid P \sqcap Q \mid P \extchoice Q \mid P \sliding Q \mid \\
	         & P | |_{A} Q \mid P \backslash A \mid f(P) \mid P \triangle Q \mid P \theta_{A} Q \mid
\end{align*}
Recall also the grammar of the language $\acptf$ as shown in Definition \ref{dfn:acptf}:
\begin{align*}
	P, Q ::= a \mid \delta \mid P . Q \mid P + Q \mid P \merge Q \mid P \leftmerge Q \mid P | Q \mid \restrict(P) \mid \abtau(P) \mid f(P) \mid
\end{align*}
\vspace{-10pt}
Finally, recall from Definitions \ref{dfn:helper-functions} and \ref{dfn:communications} the list of functions and communications:

\begin{multicols}{2}
	\begin{itemize}[leftmargin=*]
		\item $f_{\mathtt{trig}}(\alpha) = \begin{cases}
				      a_{\mathtt{ini}} & \text{if } \alpha = a_{\mathtt{first}} \\
				      a                & \text{if } \alpha = a_{\mathtt{next}}  \\
				      \alpha           & \text{otherwise}
			      \end{cases}$

		\item $f_{\mathtt{post}}(\alpha) = \begin{cases}
				      a      & \text{ if } \alpha = a_{\mathrm{post}} \\
				      \alpha & \text{otherwise}
			      \end{cases}$

		\item $f_{\mathrm{syn}}(\alpha) = \begin{cases}
				      \alpha_{\mathtt{syn}} & \text{ if } \alpha\in A \\
				      \alpha                & \text{otherwise}
			      \end{cases}$

		\item $\fdef{origin}(\alpha) = \begin{cases}
				      a_{\mathtt{origin}} & \text{if } a\in\Sigma \\
				      \tau                & \text{otherwise}
			      \end{cases}$
		\item $f_{\mathtt{split}}(\alpha) = \begin{cases}
				      a_{\mathtt{split}}  & \text{ if } a\in A                    \\
				      a_{\mathtt{origin}} & \text{ if } a\not\in A,\, a\in \Sigma \\
				      \tau                & \mathrm{otherwise}
			      \end{cases}$
	\end{itemize}
\end{multicols}
\begin{multicols}{3}
	\begin{itemize}[leftmargin=*]
		\item $a | \mathtt{first} = a_{\mathtt{first}}$\newline $a | \mathtt{next} = a_{\mathtt{next}} $
		\item $a_{\mathtt{syn}} | a_{\mathtt{syn}} = a_{\mathtt{post}}$
		\item $a_{\mathtt{ini}} | \mathtt{choose} = a_{\mathtt{post}}$
		\item $a_{\mathtt{origin}} | \mathtt{origin} = a_{\mathtt{post}}$
		\item $a_{\mathtt{ini}} | \mathtt{split} = a_{\mathtt{post}}$
		\item $a_{\mathtt{split}} | \mathtt{split} = a_{\mathtt{post}}$
	\end{itemize}
\end{multicols}

We take $\mathscr{H} = \{\mathtt{first}, \mathtt{next}, \mathtt{origin}, \mathtt{split}, \mathtt{shift}, \mathtt{choose}\}$, and therefore we have that $H_{1} = A_{0} \uplus \{\mathtt{first}, \mathtt{next}, \mathtt{origin}, \mathtt{split}\}$ as defined in \ref{dfn:sets}. We can now define our full translation:

\begin{dfn}[Translation of CSP to ACP]{dfn:trans}{}
	Let $\tcsp$ be the expressions in the language $\csp$, and $\tacp$ be expressions in the language $\acptf$. We define a translation $\tran : \tcsp \to \tacp$ defined as such:
	\begin{align*}
		\trans{STOP}           & = \delta                                                                                                                                                                                     \\
		\trans{\mathrm{div}}   & = \langle X \mid X = \tau.X \rangle                                                                                                                                                          \\
		\trans{a \to P}        & = a.\trans{P}                                                                                                                                                                                \\
		\trans{P \intchoice Q} & = \tau.\trans{P} + \tau.\trans{Q}                                                                                                                                                            \\
		\trans{P \extchoice Q} & = \partial_{H_{0}}\Bigl(\fdef{post}\Bigl[\Gamma[\trans{P}]\, \merge \,\mathtt{choose}\, \merge \,\Gamma[\trans{Q}]\Bigl]\Bigr)                                                               \\
		\trans{P \sliding Q}   & = \partial_{H_{0}}\Bigl(\tau_{\{\mathtt{shift}\}}\bigl(f_{\mathtt{post}}\Bigl[ \Gamma(\trans{P}) \merge \mathrm{choose} \merge \mathtt{shift}_{\mathtt{ini}} . \trans{Q}) \Bigr]\bigr)\Bigr) \\
		\trans{P \pcomp Q}     & = \partial_{H_{0}}\Bigl(\fdef{post}\Bigl[\fdef{syn}(\trans{P}) \merge \fdef{syn}(\trans{Q})\Bigr]\Bigr)                                                                                      \\
		\trans{P \backslash A} & = \partial_{A}{\trans{P}}                                                                                                                                                                    \\
		% \trans{\mu p.P} &= \langle X \mid X = \tau.X \rangle
		\trans{f(P)}           & = f(\trans{P})                                                                                                                                                                               \\
		\trans{P \interrupt Q} & = \partial_{H_{0}}\Bigl(\fdef{post}\Bigl[ \bigl((\fdef{origin}\trans{P}) \merge \Pi\bigr) \merge \Gamma(\trans{Q}) \Bigr]\Bigr)                                                              \\
		\trans{P \throw Q}     & = \partial_{H_{0}}\Bigl(\fdef{post} \Bigl[ (\fdef{split} \trans{P}) | | \Pi.\trans{Q} \Bigr]\Bigr)
	\end{align*}
\end{dfn}
\end{document}
