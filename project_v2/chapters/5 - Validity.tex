%! TEX root = ../hons_project.tex

\documentclass[../hons_project.tex]{subfiles}
\begin{document}
    
\[\partial_{H_{0}}\Bigl(f_{2}\Bigl[\Gamma(P)\, | | \,\mathtt{choose}\, | | \,\Gamma(Q)\Bigr]\Bigr)\]
has identical behaviour to $P + Q$

On processes with internal actions, which we will call $\mathcal{P} = \tau_{P}.a.P$ and $\mathcal{Q} = \tau_{Q}.b.Q$, where $\tau_{P}$ indicates the number of starting $\tau$ actions in the process, possibly 0, the $\tau$ cannot communicate with $\mathtt{choose}$ so $\mathtt{choose}$ will only communicate with $a$ (Since only $a$ will be labelled with $\mathtt{first}$). However, the first action of $Q$ will still have the name $b_{\mathtt{first}}$ for $b\in A$. This lets the function effectively skip the $\tau$, then perform $\mathtt{choose}$ on $a.P \,| | \,b.Q$ which matches the behaviour of $\csp$ $\square$. 

However, this translation is not strongly bisimilar. This is due to the final restriction $\partial_{H_{0}}$, which will usually restrict any stray $a_{\mathtt{first}}$ actions (and subsequent actions/processes) that try to communicate. However, since the silent step $\tau$ is not in $A$, we will get left with stray $\tau$ actions. This doesn't hold for two processes $a.P$ and $\tau.Q$, since the translation would yield the process
\[a.(\tau | | P) + \tau.(P + Q) \ne a.P + \tau.(P + Q)\]

% \begin{figure}[t!]
% 	\centering
% 	% https://q.uiver.app/#q=WzAsNyxbMCwwLCJcXGJ1bGxldCJdLFsxLDAsIlxcYnVsbGV0Il0sWzIsMCwiXFxidWxsZXQiXSxbMCwxLCJcXGJ1bGxldCJdLFsxLDEsIlxcYnVsbGV0Il0sWzIsMSwiXFxidWxsZXQiXSxbMywxLCJcXGJ1bGxldCJdLFswLDEsImEiXSxbMSwyLCJhIl0sWzMsNCwiYSJdLFs0LDUsIlxcdGF1Il0sWzUsNiwiYSJdLFswLDMsIiIsMSx7InN0eWxlIjp7ImJvZHkiOnsibmFtZSI6ImRhc2hlZCJ9LCJoZWFkIjp7Im5hbWUiOiJub25lIn19fV0sWzEsNCwiIiwxLHsic3R5bGUiOnsiYm9keSI6eyJuYW1lIjoiZGFzaGVkIn0sImhlYWQiOnsibmFtZSI6Im5vbmUifX19XSxbNSwxLCIiLDEseyJzdHlsZSI6eyJib2R5Ijp7Im5hbWUiOiJkYXNoZWQifSwiaGVhZCI6eyJuYW1lIjoibm9uZSJ9fX1dLFsyLDYsIiIsMSx7InN0eWxlIjp7ImJvZHkiOnsibmFtZSI6ImRhc2hlZCJ9LCJoZWFkIjp7Im5hbWUiOiJub25lIn19fV1d
% 	\[\begin{tikzcd}[cramped]
% 		\bullet & \bullet & \bullet \\
% 		\bullet & \bullet & \bullet & \bullet
% 		\arrow["a", from=1-1, to=1-2]
% 		\arrow[dashed, no head, from=1-1, to=2-1]
% 		\arrow["a", from=1-2, to=1-3]
% 		\arrow[dashed, no head, from=1-2, to=2-2]
% 		\arrow[dashed, no head, from=1-3, to=2-4]
% 		\arrow["a", from=2-1, to=2-2]
% 		\arrow["\tau", from=2-2, to=2-3]
% 		\arrow[dashed, no head, from=2-3, to=1-2]
% 		\arrow["a", from=2-3, to=2-4]
% 	\end{tikzcd}\]
% 	\caption{Branching Bisimilarity}
% 	\label{branching-bisim}
% \end{figure}

\vspace{-5pt}
\begin{figure}[!ht]
	\centering
% https://q.uiver.app/#q=WzAsMTMsWzAsMCwiXFxjaXJjIl0sWzQsMCwiXFxidWxsZXQiXSxbNiwwLCJcXGJ1bGxldCJdLFswLDIsIlxcYnVsbGV0Il0sWzYsMiwiXFxidWxsZXQiXSxbNCwyLCJcXGJ1bGxldCJdLFs2LDQsIlxcYnVsbGV0Il0sWzgsMiwiXFxidWxsZXQiXSxbMiwyLCJcXGJ1bGxldCJdLFsyLDQsIlxcYnVsbGV0Il0sWzQsNCwiXFxidWxsZXQiXSxbMyw0LCJcXGJ1bGxldCJdLFs4LDQsIlxcYnVsbGV0Il0sWzAsMSwiXFx0YXUiXSxbMCwzLCJhX1xcdGV4dHtmaXJzdH0iLDIseyJjb2xvdXIiOlswLDYwLDYwXX0sWzAsNjAsNjAsMV1dLFsxLDQsIlxcdGV4dHtjaG9vc2V9IiwxLHsiY29sb3VyIjpbMCw2MCw2MF19LFswLDYwLDYwLDFdXSxbMSw1LCJhX1xcdGV4dHtmaXJzdH0iLDIseyJjb2xvdXIiOlswLDYwLDYwXX0sWzAsNjAsNjAsMV1dLFs0LDYsImFfXFx0ZXh0e2ZpcnN0fSIsMCx7ImNvbG91ciI6WzAsNjAsNjBdfSxbMCw2MCw2MCwxXV0sWzUsNiwiXFx0ZXh0e2Nob29zZX0iLDEseyJjb2xvdXIiOlswLDYwLDYwXX0sWzAsNjAsNjAsMV1dLFsxLDYsImEiXSxbNCw3LCJiX1xcdGV4dHtmaXJzdH0iLDAseyJjb2xvdXIiOlswLDYwLDYwXX0sWzAsNjAsNjAsMV1dLFswLDgsIlxcdGV4dHtjaG9vc2V9IiwxLHsiY29sb3VyIjpbMCw2MCw2MF19LFswLDYwLDYwLDFdXSxbOCw5LCJhX1xcdGV4dHtmaXJzdH0iLDAseyJjb2xvdXIiOlswLDYwLDYwXX0sWzAsNjAsNjAsMV1dLFszLDksIlxcdGV4dHtjaG9vc2V9IiwxLHsiY29sb3VyIjpbMCw2MCw2MF19LFswLDYwLDYwLDFdXSxbMCw5LCJhIl0sWzIsNywiXFx0ZXh0e2Nob29zZX0iLDEseyJjb2xvdXIiOlswLDYwLDYwXX0sWzAsNjAsNjAsMV1dLFsxLDIsImJfXFx0ZXh0e2ZpcnN0fSIsMCx7ImNvbG91ciI6WzAsNjAsNjBdfSxbMCw2MCw2MCwxXV0sWzEsNywiYiJdLFs5LDExLCJcXHRhdSJdLFsxMSwxMCwiYl9cXHRleHR7Zmlyc3R9IiwwLHsiY29sb3VyIjpbMCw2MCw2MF19LFswLDYwLDYwLDFdXSxbNiwxMiwiYl9cXHRleHR7Zmlyc3R9IiwwLHsiY29sb3VyIjpbMCw2MCw2MF19LFswLDYwLDYwLDFdXSxbNywxMiwiYV9cXHRleHR7Zmlyc3R9IiwwLHsiY29sb3VyIjpbMCw2MCw2MF19LFswLDYwLDYwLDFdXV0=
\[\begin{tikzcd}[cramped]
	\circ &&&& \bullet && \bullet \\
	\\
	\bullet && \bullet && \bullet && \bullet && \bullet \\
	\\
	&& \bullet & \bullet & \bullet && \bullet && \bullet
	\arrow["\tau", from=1-1, to=1-5]
	\arrow["{a_\text{first}}"', color={rgb,255:red,214;green,92;blue,92}, from=1-1, to=3-1]
	\arrow["{\text{choose}}"{description}, color={rgb,255:red,214;green,92;blue,92}, from=1-1, to=3-3]
	\arrow["a", from=1-1, to=5-3]
	\arrow["{b_\text{first}}", color={rgb,255:red,214;green,92;blue,92}, from=1-5, to=1-7]
	\arrow["{a_\text{first}}"', color={rgb,255:red,214;green,92;blue,92}, from=1-5, to=3-5]
	\arrow["{\text{choose}}"{description}, color={rgb,255:red,214;green,92;blue,92}, from=1-5, to=3-7]
	\arrow["b", from=1-5, to=3-9]
	\arrow["a", from=1-5, to=5-7]
	\arrow["{\text{choose}}"{description}, color={rgb,255:red,214;green,92;blue,92}, from=1-7, to=3-9]
	\arrow["{\text{choose}}"{description}, color={rgb,255:red,214;green,92;blue,92}, from=3-1, to=5-3]
	\arrow["{a_\text{first}}", color={rgb,255:red,214;green,92;blue,92}, from=3-3, to=5-3]
	\arrow["{\text{choose}}"{description}, color={rgb,255:red,214;green,92;blue,92}, from=3-5, to=5-7]
	\arrow["{b_\text{first}}", color={rgb,255:red,214;green,92;blue,92}, from=3-7, to=3-9]
	\arrow["{a_\text{first}}", color={rgb,255:red,214;green,92;blue,92}, from=3-7, to=5-7]
	\arrow["{a_\text{first}}", color={rgb,255:red,214;green,92;blue,92}, from=3-9, to=5-9]
	\arrow["\tau", from=5-3, to=5-4]
	\arrow["{b_\text{first}}", color={rgb,255:red,214;green,92;blue,92}, from=5-4, to=5-5]
	\arrow["{b_\text{first}}", color={rgb,255:red,214;green,92;blue,92}, from=5-7, to=5-9]
\end{tikzcd}\]
	\caption{Counterexample for Strong Bisimilarity with the processes $P = a$ and $Q = \tau.b$. The result of the translation is $a.\tau + \tau.(a+b) \not\leftrightarroweq a + \tau.(a+b)$. Restricted actions are marked in red.}
	\label{fig:sb-counterexample}
\end{figure}


We define formally the definition of Rooted Branching Bisimulation via \citep{baetenProcessAlgebra1990}:

\begin{dfn}[Rooted Branching Bisimilarity]{dfn:rbb}{}
   Let $P$ and $Q$ be two processes, and $R$ be a relation between nodes of $P$ and nodes of $Q$. $R$ is a \textbf{Branching Bisimulation} between $P$ and $Q$ if:
   \begin{enumerate}
      \item The roots of $P$ and $Q$ are related by $R$
      \item If $s \prightarrow{a} s'$ for $a\in A \cup \{\tau\}$ is an edge in $P$, and $s R t$, then either
	 \begin{enumerate}[label=\alph*)]
	    \item $a= \tau$ and $s' R t$
	    \item $\exists t \Rightarrow t_{1} \prightarrow{a} t'$ such that $s R t_{1}$ and $s R t'$
	 \end{enumerate}
      \item If $s \downarrow$ and $s R t$ then there exists a path $t \Rightarrow t'$ in $Q$ to a node $t'$ with $t'\downarrow$ and $s R t'$
      \item[4, 5]: As in $2, 3$, with the roles of $P$ and $Q$ interchanged
   \end{enumerate}

   \longrule{0.08ex}

   $R$ is called a \textbf{Rooted Branching Bisimulation} if the following root condition is satisfied:
   \begin{itemize}
      \item If $\mathrm{root}(P) \prightarrow{a} s'$ for $a\in A \cup \{\tau\}$, then there is a $t'$ with $\mathrm{root}(Q)\prightarrow{a} t'$ and $s' R t'$
      \item If $\mathrm{root}(Q) \prightarrow{a} t'$ for $a\in A \cup \{\tau\}$, then there is a $s'$ with $\mathrm{root}(P)\prightarrow{a} s'$ and $s' R t'$
      \item $\mathrm{root}(g)\downarrow$ iff $\mathrm{root}(h)\downarrow$
   \end{itemize}
\end{dfn}

In other words, two processes are Rooted Branching Bisimilar if it is strongly bisimilar for the first step, and branching bisimilar for the remaining ones. This is a more desirable outcome because RBB is a congruence \citep{fokkinkRootedBranchingBisimulation2000}.

We can now work towards a proof that the external choice is Rooted Branching Bisimilar.

\newpage


\begin{dfn}[The translation]{dfn:the-translation}{}
\[\trans{P \square Q} = \partial_{H_{0}}\Bigl(f_{1}\Bigl[\Gamma[\trans{P}]\, | | \,\mathtt{choose}\, | | \,\Gamma[\trans{Q}]\Bigl]\Bigr)\]
\end{dfn}

\begin{dfn}[Communications]{dfn:the-comms}{}
	From \ref{dfn:comms-external}:
	\[a | \mathtt{first} = a_{\mathtt{first}} \qquad a | \mathtt{next} = a_{\mathtt{next}} \qquad a_{\mathtt{ini}} | \mathtt{choose} = a_{\mathtt{post}}\]
		
\end{dfn}


\begin{lma}[]{lma:tau-transform}{}
	For processes $P, P'\in \csp$, if $P \prightarrow{\tau} P'$ then $\Gamma[\trans{P}] \prightarrow{\tau} \Gamma[\trans{P'}]$. Alternatively,
	\[\Gamma[\trans{P}] = \tau. \Gamma[\trans{P'}]\]
\end{lma}

\begin{proof}
	Something about how $\tau$'s do not get affected by the gamma function.
\end{proof}

\begin{lma}[]{lma:a-transform}{}
	For processes $P, P'\in \csp$, if $P \prightarrow{a} P'$ then for any action $a \in A_{0}$ we have 
	\[
		(\Gamma[\trans{P}]\, | | \, \mathrm{choose}) \prightarrow{a} \trans{P'} \quad \text{ and } \quad 
		(\mathrm{choose}\, | | \, (\Gamma[\trans{P}]) \prightarrow{a} \trans{P'}
	\]
	and
\end{lma}

\begin{proof}
	Directly follows from communications
\end{proof}

\begin{lma}[]{lma:stagnant-inis}{}
	For processes $a.P\in \csp$ where $a\in A_{0}$, if there exists a process $b.P'$ where $b\in A_{0}$ we have 
	
	\[\partial_{H_{0}} (b.P' \,| | \, \Gamma[\trans{a.P}]) = \trans{b.P'} \]
\end{lma}

\begin{proof}
	The $\Gamma$ function turns a process into a process equation of the form
	\[a_{\mathtt{ini}}.b.c\cdots\]
	the process $a_{\mathtt{ini}}$ does not communicate with anything other than $\mathtt{choose}$ which is not in $A_{0}$. Therefore, the restriction operator will remove all the $\Gamma$ left merges, leaving $b.P'$
\end{proof}

\newpage
\begin{lma}[]{lma:tau-on-p-process}{}
	% For a process $\tau^{*}.a.P\in \csp$, where $\tau^{*}$ indicates a chain of $\tau$, possibly $0$, and $a\in A_{0}$, if there exists a process $Q\in \acptf$ where $b\in A_{0}$ for all actions $b$, we have 
	% 
	% \begin{align*}
	%    \partial_{H_{0}} (Q\,| | \Gamma[\trans{\tau^{*}.a.P}]) &= \partial_{H_{0}} (Q\,| | \tau^{*}\Gamma[\trans{a.P}]) \\
	% 										&=(Q\, | | \, \tau^{*})
	% \end{align*}
	% This resulting process is Branching Bisimilar to $\trans{a.P'}$

	In the language $\acptf$, a process $\tau^{*}.P$ is Rooted Branching Bisimilar to the process $P$. Here, I use the notation of $\tau^{*}$ indicating a chain of $\tau$, possibly $0$.
\end{lma}

\textbf{FROM CHRIS}
\begin{enumerate}
    \item Given $P \stackrel{a}\to P'$ there must be some operational rule of the form $\frac{P''\:\stackrel{a}\to\:P'}{P \:\stackrel{a}\to\: P'}$, which suggests that there is a $P''$ such that $P'' \stackrel{a}\to P'$.
    \item The process $P''$ has a proof shorter than that of $P$ by one step. Due to the inductive hypothesis, this means that there exists a $q''$ such that $(P'', P'') \in \mathcal B$.
    \item Because $(P'', Q'') \in \mathcal B$ and $P'' \stackrel{a}\to P'$ there must be a $Q'$ such that $Q'' \stackrel{a}\to Q'$ and $(P', Q') \in \mathcal B$.
    \item When a rule of the form $\frac{Q'' \:\stackrel{a}\to\: Q'}{Q\:\stackrel{a}\to\:Q'}$ exists it means $P \stackrel{a}\to P'$ and $(P', Q') \in \mathcal B$ implies the existance of a $Q'$ such that $Q \stackrel{a}\to Q'$, fulfilling the criteria of the bisimulation condition. 
\end{enumerate}

\begin{proof}
   From the definition of Branching Bisimilarity, processes $P$ and $Q$ are related under a relation $\brb$ if:
   \begin{enumerate}
      \item The roots of $P$ and $Q$ are related by $\brb$
      \item If $s \prightarrow{a} s'$ for $a\in A \cup \{\tau\}$ is an edge in $P$, and $s R t$, then either
	 \begin{enumerate}[label=\alph*)]
	    \item $a= \tau$ and $s' \brb t$
	    \item $\exists t \Rightarrow t_{1} \prightarrow{a} t'$ such that $s \brb t_{1}$ and $s \brb t'$
	 \end{enumerate}
      \item If $s \downarrow$ and $s \brb t$ then there exists a path $t \Rightarrow t'$ in $Q$ to a node $t'$ with $t'\downarrow$ and $s \brb t'$
      \item[4, 5]: As in $2, 3$, with the roles of $P$ and $Q$ interchanged
   \end{enumerate}
\end{proof}

\newpage
\section{Section}
\subsection{Proof of Rooted Branching Bisimilarity [WIP]}




We define a bisimulation relation.
\begin{dfn}[Rooted Branching Bisimulation Relation]{dfn:bisim-relation}{}
   Let $\tcsp$ be the expressions in the language $\csp$, and $\tacp$ be expressions in the language $\acptf$. We use the translation $\tran : \tcsp \to \tacp$ as defined in \ref{dfn:trans}. 

   We now define a Rooted Branching Bisimulation between $\tcsp$ and $\tacp$:
   \[
      \mathcal{B} := \{P, \trans{P} \mid P\in \csp\}
   \]
\end{dfn}

\begin{proof}
	
Let $P, Q \in \csp$ be two processes. WTS $P \square Q =_{\text{RBB}} \trans{P \square Q}$. I.e.: we want to show that any move will result in a process that satisfies RBB. We prove this inductively by showing that any smaller process of $P$ is also Rooted Branching Bisimilar.

Define relation 

% \textbf{Option 1}: $a$ action on $P$. Let $a \in A_{0}$ and $P$ s.t. $P \prightarrow{a} P'$. In the domain of $\csp$, this results in the process
% \[P \square Q \prightarrow{a} P'\]
% and therefore, there exists $P$ such that $P \prightarrow{a} P'$. Therefore, there exists a smaller process $P$ and by the inductive hypothesis, there exists a related process $\trans{P}$ s.t. $P \sim \trans{P}$, and hence an action $\trans{P} \prightarrow{a} \trans{P}'$ and $P' \sim \trans{P}'$. Via our relation $\mathcal{B}$, $P' \sim \trans{P'}$, therefore $\trans{P'} = \trans{P}'$
% 	
% Via Lemma \ref{lma:a-transform}, we can derive the following equation from our translation
% \[\prftree{P \prightarrow{a} P'}{\partial_{H_{0}}\Bigl(f_{1}\Bigl[\Gamma[trans)]\, | | \,\mathtt{choose}\, | | \,\Gamma[trans}]\Bigl]\Bigr} \prightarrow{a} \partial_{H_{0}}\Bigl(f_{1}\Bigl[trans'}\, | | \,\Gamma[trans)]\Bigl]\Bigr}}\]


\textbf{Option 1}: $a$ action on $P$. Let $a \in A_{0}$ and $P'\in \mathrm{CSP}$ s.t. $P \prightarrow{a} P'$. In the domain of CSP, this results in the process

		\[P \square Q \prightarrow{a} P'\]
	
	Via Lemma \ref{lma:a-transform}, we can now derive the following equation
	\[\trans{P \square Q} = \partial_{H_{0}}\Bigl(f_{1}\Bigl[\Gamma[\trans{P}]\, | | \,\mathtt{choose}\, | | \,\Gamma[\trans{Q}]\Bigl]\Bigr) \prightarrow{a} \partial_{H_{0}}\Bigl(f_{1}\Bigl[\trans{P'}\, | | \,\Gamma[\trans{Q}]\Bigl]\Bigr)\]
      




      This is not yet a process that is comparable to $\trans{P'}$, so we look at the next step.
\begin{enumerate}
	\item \textbf{Case 1}: $\exists b\in A_{0}$ and $Q'\in \csp$ s.t. $Q \prightarrow{b} Q'$. Then via Lemma \ref{lma:stagnant-inis}, this results in
		\[\trans{P'} = \partial_{H_{0}}(f_{1}[\trans{P'}]) = \trans{P'}\]
		Which is strongly bisimilar.
	\item \textbf{Case 2}: $Q \prightarrow{ \tau} Q'$. Then via Lemma \ref{lma:tau-transform}, we can derive the following equation:

	\[\mathscr{T}(P') = \partial_{H_{0}}\Bigl(f_{1}\Bigl[\mathscr{T}(P')\, | | \,\tau.\Gamma[\mathscr{T}(Q')]\Bigl]\Bigr)\]
		Case 2 can be repeated as many times as needed until $Q$ reaches Case 1. Via Lemma \ref{lma:tau-on-p-process}, this results in
		\[\trans{P'} = \partial_{H_{0}}(f_{1}[\trans{P'}\, | | \, \Rightarrow]) = (\trans{P'}\, | | \, \Rightarrow)\]
		Which is Branching Bisimilar via Lemma \ref{lma:tau-on-p-process}
\end{enumerate}

The first action $(a)$ can happen Strongly on the root, and any further actions are either also Strongly Bisimilar, or Branching Bisimilar. Therefore, an $a$ action on $P$ is Rooted Branching Bisimilar

% \noindent\rule{\textwidth}{0.08ex}

\textbf{Option 2}: $\tau$ action on $P$. Let $P'\in \csp$ s.t. $P \prightarrow{\tau} P'$. In the domain of $\csp$, this results in the process
\[P \square Q \prightarrow{\tau} P' \square Q\]

	Via Lemma \ref{lma:tau-transform}, we can now derive the following equation
	\[
		\partial_{H_{0}}\Bigl(f_{1}\Bigl[\Gamma[\trans{P}]\, | | \,\mathtt{choose}\, | | \,\Gamma[\trans{Q}]\Bigl]\Bigr) \prightarrow{\tau} \partial_{H_{0}}\Bigl(f_{1}\Bigl[\Gamma[\trans{P'}]\, | | \,\mathtt{choose}\, | | \,\Gamma[\trans{Q}]\Bigl]\Bigr)
			\]
Which is strongly bisimilar to $\trans{P' \square Q}$, therefore a $\tau$ action on $P$ is also Rooted Branching Bisimilar.

\noindent\rule{\textwidth}{0.08ex}

\textbf{Option 3, Option 4}: The same logic from option $1$ and option $2$ can be applied to $Q$ in reverse to obtain processes that satisfy Rooted Branching Bisimilarity.

\noindent\rule{\textwidth}{0.08ex}

We have now exhausted cases, and therefore can conclude that our translation of $\csp$ External Choice is Rooted Branching Bisimilar, and therefore a Congruence.
\end{proof}

\subsection{Stopping}

Similarly to external choice, this is also Rooted Branching Bisimilar.

A counterexample to the encoding being Strongly Bisimilar is with the trivial example
\[P = \tau,\,Q = b\]
This should yield the following process graph:
% https://q.uiver.app/#q=WzAsNCxbMCwwLCJcXGNpcmMiXSxbMiwwLCJcXGJ1bGxldCJdLFswLDIsIlxcYnVsbGV0Il0sWzIsMiwiXFxidWxsZXQiXSxbMCwxLCJcXHRhdSIsMix7ImNvbG91ciI6WzAsNjAsNjBdfSxbMCw2MCw2MCwxXV0sWzAsMiwiYiIsMCx7ImNvbG91ciI6WzI0MCw2MCw2MF19LFsyNDAsNjAsNjAsMV1dLFsxLDMsImIiLDAseyJjb2xvdXIiOlsyNDAsNjAsNjBdfSxbMjQwLDYwLDYwLDFdXV0=
\[\begin{tikzcd}[cramped, column sep=scriptsize]
	\circ && \bullet \\
	\\
	\bullet && \bullet
	\arrow["\tau"', color={rgb,255:red,214;green,92;blue,92}, from=1-1, to=1-3]
	\arrow["b", color={rgb,255:red,92;green,92;blue,214}, from=1-1, to=3-1]
	\arrow["b", color={rgb,255:red,92;green,92;blue,214}, from=1-3, to=3-3]
\end{tikzcd}\]

However, it yields the following

% https://q.uiver.app/#q=WzAsMTYsWzUsNCwiXFxjaXJjIl0sWzcsMywiXFxidWxsZXQiXSxbNSwzLCJcXGJ1bGxldCJdLFs3LDIsIlxcYnVsbGV0Il0sWzMsNSwiXFxidWxsZXQiXSxbMiwzLCJcXGJ1bGxldCJdLFsyLDIsIlxcYnVsbGV0Il0sWzAsNCwiXFxidWxsZXQiXSxbMyw0LCJcXGJ1bGxldCJdLFswLDMsIlxcYnVsbGV0Il0sWzksMiwiXFxidWxsZXQiXSxbOSwxLCJcXGJ1bGxldCJdLFs2LDAsIlxcYnVsbGV0Il0sWzYsMSwiXFxidWxsZXQiXSxbNCwyLCJcXGJ1bGxldCJdLFs0LDEsIlxcYnVsbGV0Il0sWzAsMSwiXFxtYXRodHR7b3JpZ2lufSIsMix7ImNvbG91ciI6WzAsMCwzOF19LFswLDAsMzgsMV1dLFswLDIsIlxcdGF1IiwyLHsiY29sb3VyIjpbMCw2MCw2MF19LFswLDYwLDYwLDFdXSxbMSwzLCJcXHRhdSIsMix7ImNvbG91ciI6WzAsNjAsNjBdfSxbMCw2MCw2MCwxXV0sWzIsMywiXFxtYXRodHR7b3JpZ2lufSIsMix7ImNvbG91ciI6WzAsMCwzOF19LFswLDAsMzgsMV1dLFswLDUsImJfe1xcbWF0aHR0e2ZpcnN0fX0iLDIseyJjb2xvdXIiOlsyNDAsNjAsNjBdLCJzdHlsZSI6eyJib2R5Ijp7Im5hbWUiOiJkYXNoZWQifX19LFsyNDAsNjAsNjAsMV1dLFsyLDYsImJfe1xcbWF0aHR0e2ZpcnN0fX0iLDIseyJjb2xvdXIiOlsyNDAsNjAsNjBdLCJzdHlsZSI6eyJib2R5Ijp7Im5hbWUiOiJkYXNoZWQifX19LFsyNDAsNjAsNjAsMV1dLFs1LDYsIlxcdGF1IiwyLHsiY29sb3VyIjpbMCw2MCw2MF19LFswLDYwLDYwLDFdXSxbMCw0LCJcXG1hdGh0dHtzcGxpdH0iLDAseyJjb2xvdXIiOlswLDAsMzhdfSxbMCwwLDM4LDFdXSxbNCw3LCJiX3tcXG1hdGh0dHtmaXJzdH19IiwyLHsiY29sb3VyIjpbMjQwLDYwLDYwXSwic3R5bGUiOnsiYm9keSI6eyJuYW1lIjoiZGFzaGVkIn19fSxbMjQwLDYwLDYwLDFdXSxbNCw4LCJcXHRhdSIsMCx7ImNvbG91ciI6WzAsNjAsNjBdfSxbMCw2MCw2MCwxXV0sWzcsOSwiXFx0YXUiLDAseyJjb2xvdXIiOlswLDYwLDYwXX0sWzAsNjAsNjAsMV1dLFs4LDksImJfe1xcbWF0aHR0e2ZpcnN0fX0iLDIseyJjb2xvdXIiOlsyNDAsNjAsNjBdLCJzdHlsZSI6eyJib2R5Ijp7Im5hbWUiOiJkYXNoZWQifX19LFsyNDAsNjAsNjAsMV1dLFsyLDgsIlxcbWF0aHR0e3NwbGl0fSIsMCx7ImNvbG91ciI6WzAsMCwzOF19LFswLDAsMzgsMV1dLFs2LDksIlxcbWF0aHR0e3NwbGl0fSIsMCx7ImNvbG91ciI6WzAsMCwzOF19LFswLDAsMzgsMV1dLFs1LDcsIlxcbWF0aHR0e3NwbGl0fSIsMCx7ImNvbG91ciI6WzAsMCwzOF19LFswLDAsMzgsMV1dLFsxLDEwLCJcXG1hdGh0dHtzcGxpdH0iLDIseyJjb2xvdXIiOlswLDAsMzhdfSxbMCwwLDM4LDFdXSxbMywxMSwiXFxtYXRodHR7c3BsaXR9IiwyLHsiY29sb3VyIjpbMCwwLDM4XX0sWzAsMCwzOCwxXV0sWzExLDEyLCJiX3tcXG1hdGh0dHtmaXJzdH19IiwyLHsiY29sb3VyIjpbMjQwLDYwLDYwXSwic3R5bGUiOnsiYm9keSI6eyJuYW1lIjoiZGFzaGVkIn19fSxbMjQwLDYwLDYwLDFdXSxbMTAsMTMsImJfe1xcbWF0aHR0e2ZpcnN0fX0iLDIseyJjb2xvdXIiOlsyNDAsNjAsNjBdLCJzdHlsZSI6eyJib2R5Ijp7Im5hbWUiOiJkYXNoZWQifX19LFsyNDAsNjAsNjAsMV1dLFsxMCwxMSwiXFx0YXUiLDIseyJjb2xvdXIiOlswLDYwLDYwXX0sWzAsNjAsNjAsMV1dLFsxMywxMiwiXFx0YXUiLDIseyJjb2xvdXIiOlswLDYwLDYwXX0sWzAsNjAsNjAsMV1dLFs1LDE0LCJcXG1hdGh0dHtvcmlnaW59IiwyLHsiY29sb3VyIjpbMCwwLDM4XX0sWzAsMCwzOCwxXV0sWzYsMTUsIlxcbWF0aHR0e29yaWdpbn0iLDIseyJjb2xvdXIiOlswLDAsMzhdfSxbMCwwLDM4LDFdXSxbMSwxNCwiYl97XFxtYXRodHR7Zmlyc3R9fSIsMix7ImNvbG91ciI6WzI0MCw2MCw2MF0sInN0eWxlIjp7ImJvZHkiOnsibmFtZSI6ImRhc2hlZCJ9fX0sWzI0MCw2MCw2MCwxXV0sWzMsMTUsImJfe1xcbWF0aHR0e2ZpcnN0fX0iLDIseyJjb2xvdXIiOlsyNDAsNjAsNjBdLCJzdHlsZSI6eyJib2R5Ijp7Im5hbWUiOiJkYXNoZWQifX19LFsyNDAsNjAsNjAsMV1dLFsxNCwxNSwiXFx0YXUiLDAseyJjb2xvdXIiOlswLDYwLDYwXX0sWzAsNjAsNjAsMV1dLFsxNSwxMiwiXFxtYXRodHR7c3BsaXR9IiwyLHsiY29sb3VyIjpbMCwwLDM4XX0sWzAsMCwzOCwxXV0sWzE0LDEzLCJcXG1hdGh0dHtzcGxpdH0iLDIseyJjb2xvdXIiOlswLDAsMzhdfSxbMCwwLDM4LDFdXSxbMCw3LCJiIiwwLHsiY29sb3VyIjpbMjQwLDYwLDYwXX0sWzI0MCw2MCw2MCwxXV0sWzIsOSwiYiIsMCx7ImNvbG91ciI6WzI0MCw2MCw2MF19LFsyNDAsNjAsNjAsMV1dLFszLDEyLCJiIiwyLHsiY29sb3VyIjpbMjQwLDYwLDYwXX0sWzI0MCw2MCw2MCwxXV0sWzEsMTMsImIiLDIseyJjb2xvdXIiOlsyNDAsNjAsNjBdfSxbMjQwLDYwLDYwLDFdXV0=
\[\begin{tikzcd}[cramped]
	&&&&&& \bullet \\
	&&&& \bullet && \bullet &&& \bullet \\
	&& \bullet && \bullet &&& \bullet && \bullet \\
	\bullet && \bullet &&& \bullet && \bullet \\
	\bullet &&& \bullet && \circ \\
	&&& \bullet
	\arrow["{\mathtt{split}}"', color={rgb,255:red,97;green,97;blue,97}, from=2-5, to=1-7]
	\arrow["\tau"', color={rgb,255:red,214;green,92;blue,92}, from=2-7, to=1-7]
	\arrow["{b_{\mathtt{first}}}"', color={rgb,255:red,92;green,92;blue,214}, dashed, from=2-10, to=1-7]
	\arrow["{\mathtt{origin}}"', color={rgb,255:red,97;green,97;blue,97}, from=3-3, to=2-5]
	\arrow["{\mathtt{split}}", color={rgb,255:red,97;green,97;blue,97}, from=3-3, to=4-1]
	\arrow["\tau", color={rgb,255:red,214;green,92;blue,92}, from=3-5, to=2-5]
	\arrow["{\mathtt{split}}"', color={rgb,255:red,97;green,97;blue,97}, from=3-5, to=2-7]
	\arrow["b"', color={rgb,255:red,92;green,92;blue,214}, from=3-8, to=1-7]
	\arrow["{b_{\mathtt{first}}}"', color={rgb,255:red,92;green,92;blue,214}, dashed, from=3-8, to=2-5]
	\arrow["{\mathtt{split}}"', color={rgb,255:red,97;green,97;blue,97}, from=3-8, to=2-10]
	\arrow["{b_{\mathtt{first}}}"', color={rgb,255:red,92;green,92;blue,214}, dashed, from=3-10, to=2-7]
	\arrow["\tau"', color={rgb,255:red,214;green,92;blue,92}, from=3-10, to=2-10]
	\arrow["\tau"', color={rgb,255:red,214;green,92;blue,92}, from=4-3, to=3-3]
	\arrow["{\mathtt{origin}}"', color={rgb,255:red,97;green,97;blue,97}, from=4-3, to=3-5]
	\arrow["{\mathtt{split}}", color={rgb,255:red,97;green,97;blue,97}, from=4-3, to=5-1]
	\arrow["{b_{\mathtt{first}}}"', color={rgb,255:red,92;green,92;blue,214}, dashed, from=4-6, to=3-3]
	\arrow["{\mathtt{origin}}"', color={rgb,255:red,97;green,97;blue,97}, from=4-6, to=3-8]
	\arrow["b", color={rgb,255:red,92;green,92;blue,214}, from=4-6, to=4-1]
	\arrow["{\mathtt{split}}", color={rgb,255:red,97;green,97;blue,97}, from=4-6, to=5-4]
	\arrow["b"', color={rgb,255:red,92;green,92;blue,214}, from=4-8, to=2-7]
	\arrow["{b_{\mathtt{first}}}"', color={rgb,255:red,92;green,92;blue,214}, dashed, from=4-8, to=3-5]
	\arrow["\tau"', color={rgb,255:red,214;green,92;blue,92}, from=4-8, to=3-8]
	\arrow["{\mathtt{split}}"', color={rgb,255:red,97;green,97;blue,97}, from=4-8, to=3-10]
	\arrow["\tau", color={rgb,255:red,214;green,92;blue,92}, from=5-1, to=4-1]
	\arrow["{b_{\mathtt{first}}}"', color={rgb,255:red,92;green,92;blue,214}, dashed, from=5-4, to=4-1]
	\arrow["{b_{\mathtt{first}}}"', color={rgb,255:red,92;green,92;blue,214}, dashed, from=5-6, to=4-3]
	\arrow["\tau"', color={rgb,255:red,214;green,92;blue,92}, from=5-6, to=4-6]
	\arrow["{\mathtt{origin}}"', color={rgb,255:red,97;green,97;blue,97}, from=5-6, to=4-8]
	\arrow["b", color={rgb,255:red,92;green,92;blue,214}, from=5-6, to=5-1]
	\arrow["{\mathtt{split}}", color={rgb,255:red,97;green,97;blue,97}, from=5-6, to=6-4]
	\arrow["{b_{\mathtt{first}}}"', color={rgb,255:red,92;green,92;blue,214}, dashed, from=6-4, to=5-1]
	\arrow["\tau", color={rgb,255:red,214;green,92;blue,92}, from=6-4, to=5-4]
\end{tikzcd}\]

Which reduces down to:
% https://q.uiver.app/#q=WzAsNCxbMCwwLCJcXGNpcmMiXSxbMiwwLCJcXGJ1bGxldCJdLFswLDIsIlxcYnVsbGV0Il0sWzIsMiwiXFxidWxsZXQiXSxbMCwxLCJcXHRhdSIsMix7ImNvbG91ciI6WzAsNjAsNjBdfSxbMCw2MCw2MCwxXV0sWzIsMywiXFx0YXUiLDAseyJjb2xvdXIiOlswLDYwLDYwXX0sWzAsNjAsNjAsMV1dLFswLDIsImIiLDAseyJjb2xvdXIiOlsyNDAsNjAsNjBdfSxbMjQwLDYwLDYwLDFdXSxbMSwzLCJiIiwwLHsiY29sb3VyIjpbMjQwLDYwLDYwXX0sWzI0MCw2MCw2MCwxXV1d
\[\begin{tikzcd}[cramped, column sep=scriptsize]
	\circ && \bullet \\
	\\
	\bullet && \bullet
	\arrow["\tau"', color={rgb,255:red,214;green,92;blue,92}, from=1-1, to=1-3]
	\arrow["b", color={rgb,255:red,92;green,92;blue,214}, from=1-1, to=3-1]
	\arrow["b", color={rgb,255:red,92;green,92;blue,214}, from=1-3, to=3-3]
	\arrow["\tau", color={rgb,255:red,214;green,92;blue,92}, from=3-1, to=3-3]
\end{tikzcd}\]

\subsection{Generalising}

\begin{lma}[Maybe?]{lma:maybe-prove}{}
   I claim that a Strong Bisimulation cannot occur and RBB is the finest equivalence able to be translated
\end{lma}

\begin{proof}
   Something about how taus cannot be renamed / communicated. Have not thought of it past there but it feels like it might be right as all the translations have the same issue.
\end{proof}
\end{document}
