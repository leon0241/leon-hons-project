%! TEX root = ../hons_project.tex

\documentclass[../hons_project.tex]{subfiles}
\begin{document}

Referring back to our translation \ref{dfn:trans}, the translation for External Choice
\[\partial_{H_{0}}(\fdef{post}[\Gamma(P)\merge\mathtt{choose}\merge\Gamma(Q)])\]
has identical behaviour to $P + Q$ when dealing with processes with only external actions. However, on processes with internal actions, the translation is not so trivial. The addition of an internal action works in the translation's favour for actions such as deferring a $\mathtt{start}$ tag to the first visible action (see \ref{fig:triggering-tau}), but it can also backfire, as the translation largely relies on removing unwanted left-merges and communications through restriction operators. As internal actions are non-interactable it means restrictions that \textit{should} remove all remaining actions will end up with unwanted left-over $\tau$ moves.
\newline For example, if we take the process $a \extchoice \tau.b\in \csp$, the resulting translation in $\acptf$ is
\vspace{-5pt}
\[\partial_{H_{0}}\Bigl(\fdef{post}\Bigl[\Gamma(a) \merge \mathtt{choose} \merge \Gamma(\tau.b)\Bigr]\Bigr) = \partial_{H_{0}}\Bigl(\fdef{post}\Bigl[a_{\mathtt{ini}} \merge \mathtt{choose} \merge \tau.b_{\mathtt{ini}}\Bigr]\Bigr)\]
which has the following process graph:

\begin{figure}[!hb!]
	\centering
	% https://q.uiver.app/#q=WzAsMTIsWzAsMywiXFxidWxsZXQiXSxbMiwyLCJcXGJ1bGxldCJdLFsyLDUsIlxcYnVsbGV0Il0sWzQsMSwiXFxidWxsZXQiXSxbNCw0LCJcXGJ1bGxldCJdLFs2LDMsIlxcYnVsbGV0Il0sWzAsMiwiXFxidWxsZXQiXSxbMiwxLCJcXGJ1bGxldCJdLFs0LDAsIlxcYnVsbGV0Il0sWzYsMiwiXFxidWxsZXQiXSxbNCwzLCJcXGJ1bGxldCJdLFsyLDQsIlxcYnVsbGV0Il0sWzAsMSwiXFx0YXUiLDFdLFswLDIsImFfe1xcbWF0aHR0e2luaX19IiwxLHsiY29sb3VyIjpbMjQwLDYwLDc2XSwic3R5bGUiOnsiYm9keSI6eyJuYW1lIjoiZGFzaGVkIn19fSxbMjQwLDYwLDc2LDFdXSxbMSwzLCJiX3tcXG1hdGh0dHtpbml9fSIsMSx7ImNvbG91ciI6WzAsNjAsNzFdLCJzdHlsZSI6eyJib2R5Ijp7Im5hbWUiOiJkYXNoZWQifX19LFswLDYwLDcxLDFdXSxbMSw0LCJhX3tcXG1hdGh0dHtpbml9fSIsMSx7ImNvbG91ciI6WzI0MCw2MCw3Nl0sInN0eWxlIjp7ImJvZHkiOnsibmFtZSI6ImRhc2hlZCJ9fX0sWzI0MCw2MCw3NiwxXV0sWzMsNSwiYV97XFxtYXRodHR7aW5pfX0iLDEseyJjb2xvdXIiOlsyNDAsNjAsNzZdLCJzdHlsZSI6eyJib2R5Ijp7Im5hbWUiOiJkYXNoZWQifX19LFsyNDAsNjAsNzYsMV1dLFsyLDQsIlxcdGF1IiwyLHsiY29sb3VyIjpbMCwwLDQ2XX0sWzAsMCw0NiwxXV0sWzQsNSwiYl97XFxtYXRodHR7aW5pfX0iLDEseyJjb2xvdXIiOlswLDYwLDcxXSwic3R5bGUiOnsiYm9keSI6eyJuYW1lIjoiZGFzaGVkIn19fSxbMCw2MCw3MSwxXV0sWzAsNiwiXFxtYXRodHR7Y2hvb3NlfSIsMCx7ImNvbG91ciI6WzAsMCw0MF0sInN0eWxlIjp7ImJvZHkiOnsibmFtZSI6ImRhc2hlZCJ9fX0sWzAsMCw0MCwxXV0sWzEsNywiXFxtYXRodHR7Y2hvb3NlfSIsMCx7ImNvbG91ciI6WzAsMCw0MF0sInN0eWxlIjp7ImJvZHkiOnsibmFtZSI6ImRhc2hlZCJ9fX0sWzAsMCw0MCwxXV0sWzMsOCwiXFxtYXRodHR7Y2hvb3NlfSIsMix7ImNvbG91ciI6WzAsMCw0MF0sInN0eWxlIjp7ImJvZHkiOnsibmFtZSI6ImRhc2hlZCJ9fX0sWzAsMCw0MCwxXV0sWzUsOSwiXFxtYXRodHR7Y2hvb3NlfSIsMix7ImNvbG91ciI6WzAsMCw0MF0sInN0eWxlIjp7ImJvZHkiOnsibmFtZSI6ImRhc2hlZCJ9fX0sWzAsMCw0MCwxXV0sWzQsMTAsIlxcbWF0aHR0e2Nob29zZX0iLDIseyJjb2xvdXIiOlswLDAsNDBdLCJzdHlsZSI6eyJib2R5Ijp7Im5hbWUiOiJkYXNoZWQifX19LFswLDAsNDAsMV1dLFsyLDExLCJcXG1hdGh0dHtjaG9vc2V9IiwyLHsiY29sb3VyIjpbMCwwLDQwXSwic3R5bGUiOnsiYm9keSI6eyJuYW1lIjoiZGFzaGVkIn19fSxbMCwwLDQwLDFdXSxbMTEsMTAsIlxcdGF1IiwxXSxbMTAsOSwiYl97XFxtYXRodHR7aW5pfX0iLDEseyJjb2xvdXIiOlswLDYwLDcxXSwic3R5bGUiOnsiYm9keSI6eyJuYW1lIjoiZGFzaGVkIn19fSxbMCw2MCw3MSwxXV0sWzcsOCwiYl97XFxtYXRodHR7aW5pfX0iLDEseyJjb2xvdXIiOlswLDYwLDcxXSwic3R5bGUiOnsiYm9keSI6eyJuYW1lIjoiZGFzaGVkIn19fSxbMCw2MCw3MSwxXV0sWzYsNywiXFx0YXUiLDEseyJjb2xvdXIiOlswLDAsNDZdfSxbMCwwLDQ2LDFdXSxbNiwxMSwiYV97XFxtYXRodHR7aW5pfX0iLDEseyJjb2xvdXIiOlsyNDAsNjAsNzZdLCJzdHlsZSI6eyJib2R5Ijp7Im5hbWUiOiJkYXNoZWQifX19LFsyNDAsNjAsNzYsMV1dLFs3LDEwLCJhX3tcXG1hdGh0dHtpbml9fSIsMSx7ImNvbG91ciI6WzI0MCw2MCw3Nl0sInN0eWxlIjp7ImJvZHkiOnsibmFtZSI6ImRhc2hlZCJ9fX0sWzI0MCw2MCw3NiwxXV0sWzgsOSwiYV97XFxtYXRodHR7aW5pfX0iLDEseyJjb2xvdXIiOlsyNDAsNjAsNzZdLCJzdHlsZSI6eyJib2R5Ijp7Im5hbWUiOiJkYXNoZWQifX19LFsyNDAsNjAsNzYsMV1dLFswLDExLCJhIiwxLHsiY29sb3VyIjpbMjQwLDYwLDYwXX0sWzI0MCw2MCw2MCwxXV0sWzEsMTAsImEiLDEseyJjb2xvdXIiOlsyNDAsNjAsNjBdfSxbMjQwLDYwLDYwLDFdXSxbMyw5LCJhIiwxLHsiY29sb3VyIjpbMjQwLDYwLDc5XX0sWzI0MCw2MCw3OSwxXV0sWzEsOCwiYiIsMSx7ImNvbG91ciI6WzAsNjAsNjBdfSxbMCw2MCw2MCwxXV1d
	\[\begin{tikzcd}[cramped, row sep=scriptsize]
			&&&& \bullet \\
			&& \bullet && \bullet \\
			\bullet && \bullet &&&& \bullet \\
			\bullet &&&& \bullet && \bullet \\
			&& \bullet && \bullet \\
			&& \bullet
			\arrow["{a_{\mathtt{ini}}}"{description}, color={rgb,255:red,157;green,157;blue,231}, dashed, from=1-5, to=3-7]
			\arrow["{b_{\mathtt{ini}}}"{description}, color={rgb,255:red,225;green,137;blue,137}, dashed, from=2-3, to=1-5]
			\arrow["{a_{\mathtt{ini}}}"{description}, color={rgb,255:red,157;green,157;blue,231}, dashed, from=2-3, to=4-5]
			\arrow["{\mathtt{choose}}"', color={rgb,255:red,102;green,102;blue,102}, dashed, from=2-5, to=1-5]
			\arrow["a"{description}, color={rgb,255:red,169;green,169;blue,234}, from=2-5, to=3-7]
			\arrow["{a_{\mathtt{ini}}}"{description}, color={rgb,255:red,157;green,157;blue,231}, dashed, from=2-5, to=4-7]
			\arrow["\tau"{description}, color={rgb,255:red,117;green,117;blue,117}, from=3-1, to=2-3]
			\arrow["{a_{\mathtt{ini}}}"{description}, color={rgb,255:red,157;green,157;blue,231}, dashed, from=3-1, to=5-3]
			\arrow["b"{description}, color={rgb,255:red,214;green,92;blue,92}, from=3-3, to=1-5]
			\arrow["{\mathtt{choose}}", color={rgb,255:red,102;green,102;blue,102}, dashed, from=3-3, to=2-3]
			\arrow["{b_{\mathtt{ini}}}"{description}, color={rgb,255:red,225;green,137;blue,137}, dashed, from=3-3, to=2-5]
			\arrow["a"{description}, color={rgb,255:red,92;green,92;blue,214}, from=3-3, to=4-5]
			\arrow["{a_{\mathtt{ini}}}"{description}, color={rgb,255:red,157;green,157;blue,231}, dashed, from=3-3, to=5-5]
			\arrow["{\mathtt{choose}}", color={rgb,255:red,102;green,102;blue,102}, dashed, from=4-1, to=3-1]
			\arrow["\tau"{description}, from=4-1, to=3-3]
			\arrow["a"{description}, color={rgb,255:red,92;green,92;blue,214}, from=4-1, to=5-3]
			\arrow["{a_{\mathtt{ini}}}"{description}, color={rgb,255:red,157;green,157;blue,231}, dashed, from=4-1, to=6-3]
			\arrow["{b_{\mathtt{ini}}}"{description}, color={rgb,255:red,225;green,137;blue,137}, dashed, from=4-5, to=3-7]
			\arrow["{\mathtt{choose}}"', color={rgb,255:red,102;green,102;blue,102}, dashed, from=4-7, to=3-7]
			\arrow["\tau"{description}, from=5-3, to=4-5]
			\arrow["{\mathtt{choose}}"', color={rgb,255:red,102;green,102;blue,102}, dashed, from=5-5, to=4-5]
			\arrow["{b_{\mathtt{ini}}}"{description}, color={rgb,255:red,225;green,137;blue,137}, dashed, from=5-5, to=4-7]
			\arrow["{\mathtt{choose}}"', color={rgb,255:red,102;green,102;blue,102}, dashed, from=6-3, to=5-3]
			\arrow["\tau"', color={rgb,255:red,117;green,117;blue,117}, from=6-3, to=5-5]
		\end{tikzcd}\]
	\caption{Counterexample for Strong Bisimilarity with the processes $P = a$ and $Q = \tau.b$. The result of the translation is $a.\tau + \tau.(a+b) \not\leftrightarroweq a + \tau.(a+b)$}
	\label{fig:sb-counterexample}
\end{figure}

\newpage
\section{Prerequisites and Helper Theorems}
\subsection{Rooted Branching Bisimilarity}

Recall from Definition \ref{dfn:rooted-branching} the definition of Branching and Rooted Branching Bisimilarity:

A relation $R$ is a \textbf{Branching Bisimulation} between $P$ and $Q$ if:
	\begin{enumerate}
		\item The roots of $P$ and $Q$ are related by $R$
		\item If $s \prightarrow{\alpha} s'$ for $\alpha\in \Sigma_{\tau}$ is an edge in $P$, and $s R t$, then either
		      \begin{enumerate}[label=\alph*)]
			      \item $\alpha= \tau$ and $s' R t$
			      \item $\exists t \Rightarrow t_{1} \prightarrow{\alpha} t'$ such that $s R t_{1}$ and $s R t'$
		      \end{enumerate}
		\item If $s \checkmark$ and $s R t$ then there exists a path $t \Rightarrow t'$ in $Q$ to a node $t'$ with $t'\checkmark$ and $s R t'$\footnote{\label)As explained in Section \ref{ssec:termination}, this rule can be omitted without loss in function over the translation.}
		\item[4, 5]: As in $2, 3$, with the roles of $P$ and $Q$ interchanged
	\end{enumerate}

	\longrule{0.08ex}

	$R$ is called a \textbf{Rooted Branching Bisimulation} if the following root condition is also satisfied:
	\begin{itemize}
		\item If $\mathrm{root}(P) \prightarrow{\alpha} s'$ for $\alpha\in \Sigma_{\tau}$, then there is a $t'$ with $\mathrm{root}(Q)\prightarrow{\alpha} t'$ and $s' R t'$
		\item If $\mathrm{root}(Q) \prightarrow{\alpha} t'$ for $\alpha\in \Sigma_{\tau}$, then there is an $s'$ with $\mathrm{root}(P)\prightarrow{\alpha} s'$ and $s' R t'$
		\item $\mathrm{root}(P)\checkmark$ iff $\mathrm{root}(Q)\checkmark$
	\end{itemize}

\subsection{Lemmas and Theorems}

\begin{lma}[]{lma:tau-transform}{}
	For a trace in a process $P\in \acptf$, if there exists $P'$ such that $P \prightarrow{\tau} P'$ then application of the Triggering operator shown in \ref{dfn:acp-triggering} can be applied as $\Gamma[P] \prightarrow{\tau} \Gamma[P']$. Alternatively,
	\[\Gamma[P] = \tau. \Gamma[P']\]
\end{lma}

\begin{proof}
	We have that $P = \tau.P'$ and therefore $\Gamma[P] = \Gamma[\tau.P']$. However, since internal actions cannot interact with anything inside the triggering operator, the resulting function is the same as if the $\tau$ was instead outside the function. A diagram is provided in \ref{fig:triggering-tau}.
\end{proof}

\begin{lma}[]{lma:a-transform}{}
	For a process $P\in \acptf$, if there exists $P'$ such that $P \prightarrow{a} P'$ then for any action $a \in A_{0}$ we have
	\[
		(\Gamma[P] \merge \mathtt{choose}) \prightarrow{a} P' \quad \text{ and } \quad
		(\mathtt{choose} \merge (\Gamma[P]) \prightarrow{a} P'
	\]
\end{lma}

\begin{proof}
	Directly follows from communications
\end{proof}

\begin{lma}[]{lma:stagnant-inis}{}
	For processes $a.P\in \acptf$ where $a\in A_{0}$, if $\exists b.Q$ where $b\in A_{0}$ we have
	\[\partial_{H_{0}} (b.Q \merge \Gamma[a.P]) = b.Q \]
	\vspace{-25pt}
	\longrule{0.08ex}

	Alternatively, for processes $\taus.P\in \acptf$, where $\taus$ indicates a chain of consecutive $\tau$, possibly $0$, if there exists a process $b.Q$ where $b\in A_{0}$ we have
	\[\partial_{H_{0}} (b.Q \merge \taus.\Gamma[P]) = b.Q \merge \taus\]
\end{lma}

\begin{proof}
	The Triggering operator $\Gamma$ turns a trace into a renamed trace of the form
	\[a_{\mathtt{ini}}.b.c\cdots\]
	The process $a_{\mathtt{ini}}$ does not communicate with anything other than $\mathtt{choose}$ which is not in $A_{0}$. Therefore, the restriction operator will remove all the $\Gamma$ left merges, leaving $b.Q$. However, if there are internal actions that cannot interact with restrictions and communications, then the final process will be a communication with any remaining internal actions before the first cut off external action.
\end{proof}

\begin{lma}[]{lma:tau-on-p-process}{}
	For a process $P\in \csp$ where $\trans{P}\in\acptf$ is strongly bisimilar to $P$, a process $\taus\merge\trans{P}$ is Branching Bisimilar to the process $P$. Here, I use the notation of $\taus$ indicating a chain of $\tau$, possibly $0$.
\end{lma}

\begin{proof}
	just works
\end{proof}

We can now work towards a proof that our translation is valid up to Rooted Branching Bisimilarity.


\newpage
\section{Proof of Rooted Branching Bisimilarity}

We define a bisimulation relation.
\begin{dfn}[Rooted Branching Bisimulation Relation]{dfn:bisim-relation}{}
	Let $\tcsp$ be the expressions in the language $\csp$, and $\tacp$ be expressions in the language $\acptf$. We use the translation $\tran : \tcsp \to \tacp$ as defined in \ref{dfn:trans}.

	We now define a Rooted Branching Bisimulation between $\tcsp$ and $\tacp$:
	\[
		\rbrb\, := \{(P, \trans{P}) \mid P\in \csp\}
	\]
\end{dfn}

\subsection{External choice}

From subsection \ref{ssec:external-choice}, our translation of external choice is:
\[\trans{P \extchoice Q} = \partial_{H_{0}}\Bigl(\fdef{post}\Bigl[\Gamma[\trans{P}]\, \pcomp \,\mathtt{choose}\, \pcomp \,\Gamma[\trans{Q}]\Bigl]\Bigr)\]

\begin{proof}

	Let $P, Q \in \csp$ be two processes. We want to show that $P \extchoice Q \rbrb \trans{P \extchoice Q}$. i.e.: we want to show that any move will result in a process that satisfies RBB. We show this by exhausting all possible moves that $P \extchoice Q$ can take, and confirm that $\trans{P \extchoice Q}$ can also take them, up to Rooted Branching Bisimilarity.

	\textbf{Case 1}: $a$ action on $P$. Let $a \in A_{0}$ and $P'$ such that $P \prightarrow{a} P'$. In the domain of CSP, this results in the process
	\[P \square Q \prightarrow{a} P'\]
	Now working in $\acptf$, we want to show that the translation is valid up to RBB, i.e. $\trans{P'} \rbrb P'$. Via Lemma \ref{lma:a-transform}, we can derive the following equation
	\begin{align*}
		\trans{P \extchoice Q} = & \, \partial_{H_{0}}\Bigl(\fdef{post}\Bigl[\Gamma[\trans{P}]\merge \mathtt{choose}\merge \Gamma[\trans{Q}]\Bigl]\Bigr)\prightarrow{a} \\
		                         & \,\partial_{H_{0}}\Bigl(\fdef{post}\Bigl[\trans{P'}\merge\Gamma[\trans{Q}]\Bigl]\Bigr)
	\end{align*}

	This is not yet a process that is comparable to $P'$, so we look at the next step. Due to the Triggering operator $\Gamma$ being applied to $Q$, the only communicatable action of a trace $Q_{c}$ of $Q$ will be one tagged with an \texttt{ini}, with some number of $\tau$ actions behind it. Due to the restriction of $H_{0}$, since a \texttt{ini}-tagged operator can only communicate with the action \texttt{choose}, any of the actions past $\taus$ will get restricted, leaving
	\[
		\partial_{H_{0}}\Bigl(\fdef{post}\Bigl[\trans{P'}\merge\Gamma[\trans{Q_{c}}]\Bigl]\Bigr) \implies \partial_{H_{0}}\Bigl(\fdef{post}\Bigl[\trans{P'}\merge\taus\Bigl]\Bigr) \implies \trans{P'}\merge\taus
	\]
	for every trace $Q_{c}$ in $Q$. Via Lemma \ref{lma:tau-on-p-process}, this process is Branching Bisimilar to the process $\trans{P'}$. From this, we can see the union of every branch in $Q$ is at coarsest Branching Bisimilar, and therefore as the first action is related Strongly the process is valid up to Rooted Branching Bisimilarity when taking an $a$ action on $P$.


	\textbf{Case 2}: $\tau$ action on $P$. Let $P'$ such that $P \prightarrow{\tau} P'$. In the domain of $\csp$, this results in the process
	\[P \extchoice Q \prightarrow{\tau} P' \extchoice Q\]

	Now working in $\acptf$, we want to show that the translation is valid up to RBB, i.e. $\trans{P'\extchoice Q} \rbrb P'\extchoice Q$. Via Lemma \ref{lma:tau-transform}, we can now derive the following equation
	\begin{align*}
		 & \partial_{H_{0}}\Bigl(f_{1}\Bigl[\Gamma[\trans{P}]\, | | \,\mathtt{choose}\, | | \,\Gamma[\trans{Q}]\Bigl]\Bigr) \prightarrow{\tau} \\ &\partial_{H_{0}}\Bigl(f_{1}\Bigl[\Gamma[\trans{P'}]\, | | \,\mathtt{choose}\, | | \,\Gamma[\trans{Q}]\Bigl]\Bigr)
	\end{align*}
	Which is strongly bisimilar to $P' \extchoice Q$, therefore a $\tau$ action on $P$ is also valid up to Rooted Branching Bisimilarity.

	\noindent\rule{\textwidth}{0.08ex}

	\textbf{Case 3, Case 4}: The same logic from option $1$ and option $2$ can be applied to $Q$ and $P$ to obtain processes that satisfy Rooted Branching Bisimilarity.

	\noindent\rule{\textwidth}{0.08ex}

	We have now exhausted all cases, and therefore can conclude that our translation of $\csp$ External Choice is Rooted Branching Bisimilar, and therefore Compositional.
\end{proof}

\subsection{Interrupt}

Similarly to external choice, this is also Rooted Branching Bisimilar. We prove this similarly to the previous operator. From Definitions \ref{dfn:helper-functions}, \ref{dfn:communications}, and \ref{dfn:trans}, we recall our translation of the Interrupt operator:
\[\trans{P \triangle Q} = \partial_{H_{0}}\Bigl(f_{\mathtt{post}}\Bigl[ \bigl(f_{\mathtt{origin}}(\trans{P}) \merge \Pi\bigr) \merge \Gamma(\trans{Q}) \Bigr]\Bigr)\]

\begin{proof}
	Let $P, Q \in \csp$ be two processes. We want to show $P \interrupt Q \rbrb \trans{P \interrupt Q}$. i.e.: we want to show that any move will result in a process that satisfies RBB. We show this by exhausting all possible moves that $P \interrupt Q$ can take, and confirm that $\trans{P \interrupt Q}$ can also take them, up to Rooted Branching Bisimilarity.

	Still working..

	% \textbf{Step 1}: The first step is a communication between $\trans{P}$ with the function $\fdef{origin}$ applied to it, and $\Pi$. Recall that $\Pi$ is the process
	% \[\Pi = \langle X \mid X = \mathtt{origin}.X + \mathtt{split} \rangle\]
	% This action will either be of the form
	% \[P \prightarrow{a} P' \text{ or } P \prightarrow{\tau} \frac{P}{}\]
	%
	% Since no actions in $\fdef{origin}(\trans{P})$ will be able to synchronise with the action $\mathtt{split}$, the only other option is the first action of $\fdef{origin}(\trans{P})$ synchronising with $\mathtt{origin}$. This results in the first action of the resulting function either being an action $a$ (renamed $a_{\mathtt{post}}$, and then renamed back to $a$), or an internal action $\tau$. Either option is Strongly Bisimilar to the moves of $P \interrupt Q$. We now need to show that the next steps in the translation are Branching Bisimilar
	% \[\trans{P' \triangle Q} = \partial_{H_{0}}\Bigl(f_{\mathtt{post}}\Bigl[ \bigl(f_{\mathtt{origin}}(\trans{P'}) \merge \Pi\bigr) \merge \Gamma(\trans{Q}) \Bigr]\Bigr)\]


\end{proof}

\subsection{Generalising}

\begin{lma}[Maybe?]{lma:maybe-prove}{}
	I claim that a Strong Bisimulation cannot occur and RBB is the finest equivalence able to be translated
\end{lma}

\begin{thm}[Maybe 2]{thm:tau-comm}{}
	The following diagram cannot be modelled in $\acp$ via Communication of two processes $a$ and $\tau$
	% https://q.uiver.app/#q=WzAsMyxbMCwwLCJcXGJ1bGxldCJdLFsyLDAsIlxcYnVsbGV0Il0sWzAsMiwiXFxidWxsZXQiXSxbMCwxLCJcXHRhdSJdLFswLDIsImEiLDJdXQ==
	\[\begin{tikzcd}[cramped]
			\bullet && \bullet \\
			\\
			\bullet
			\arrow["\tau", from=1-1, to=1-3]
			\arrow["a"', from=1-1, to=3-1]
		\end{tikzcd}\]
\end{thm}

% \begin{proof}
%
% \end{proof}
\end{document}





% Axiomatic Expansion of thingy

% \begin{align*}
% \partial_{H_{0}}\Bigl(\fdef{post}\Bigl[\Gamma(a) \merge \mathtt{choose} \merge \Gamma(\tau.b)\Bigr]\Bigr) \\
% \partial_{H_{0}}\Bigl(\fdef{post}\Bigl[a_{\mathtt{first}} \merge \mathtt{choose} \merge \tau.b_{\mathtt{first}}\Bigr]\Bigr) \\
% \partial_{H_{0}}\Bigl(\fdef{post}\Bigl[(a_{\mathtt{first}} . \mathtt{choose} + \mathtt{choose} . a_{\mathtt{first}} + \mathtt{choose} \mid a_{\mathtt{first}}) \merge \tau.b_{\mathtt{first}}\Bigr]\Bigr) \\
% \partial_{H_{0}}\Bigl(\fdef{post}\Bigl[(a_{\mathtt{first}} . \mathtt{choose} + \mathtt{choose} . a_{\mathtt{first}} + a) \merge \tau.b_{\mathtt{first}}\Bigr]\Bigr) \\
% \partial_{H_{0}}\Bigl(\fdef{post}\Bigl[(a_{\mathtt{first}} . \mathtt{choose}  + \mathtt{choose} . a_{\mathtt{first}} + a) \leftmerge \tau.b_{\mathtt{first}} + \\
% \tau.b_{\mathtt{first}} \leftmerge (a_{\mathtt{first}} . \mathtt{choose}  + \mathtt{choose} . a_{\mathtt{first}} + a)+\\
% (a_{\mathtt{first}} . \mathtt{choose}  + \mathtt{choose} . a_{\mathtt{first}} + a) \mid \tau.b_{\mathtt{first}}\Bigr]\Bigr) \\
% \partial_{H_{0}}\Bigl(\fdef{post}\Bigl[{\color{red}(a_{\mathtt{first}} . \mathtt{choose} \leftmerge \tau.b_{\mathtt{first}} + \mathtt{choose} . a_{\mathtt{first}}\leftmerge \tau.b_{\mathtt{first}}} + a\leftmerge \tau.b_{\mathtt{first}}) + \\
% \tau.(b_{\mathtt{first}} \merge (a_{\mathtt{first}} . \mathtt{choose}  + \mathtt{choose} . a_{\mathtt{first}} + a))+\\
% (a_{\mathtt{first}} . \mathtt{choose} \mid \tau.b_{\mathtt{first}} + \mathtt{choose} . a_{\mathtt{first}}\mid \tau.b_{\mathtt{first}} + a\mid \tau.b_{\mathtt{first}} \Bigr]\Bigr) \\
% \partial_{H_{0}}\Bigl(\fdef{post}\Bigl[a. \tau.b_{\mathtt{first}}) + \\
% \tau.(b_{\mathtt{first}} \leftmerge (a_{\mathtt{first}} . \mathtt{choose}  + \mathtt{choose} . a_{\mathtt{first}} + a) + \\
% (a_{\mathtt{first}} . \mathtt{choose}  + \mathtt{choose} . a_{\mathtt{first}} + a)\leftmerge b_{\mathtt{first}} +\\
% b_{\mathtt{first}} \mid (a_{\mathtt{first}} . \mathtt{choose}  + \mathtt{choose} . a_{\mathtt{first}} + a)) \\
% (a_{\mathtt{first}} . \mathtt{choose} \mid b_{\mathtt{first}} + \mathtt{choose} . a_{\mathtt{first}}\mid b_{\mathtt{first}} + a\mid b_{\mathtt{first}} \Bigr]\Bigr) \\
% \end{align*}
