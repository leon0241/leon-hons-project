\documentclass[../hons_project.tex]{subfiles}
\begin{document}
    
\section{Languages and Expressiveness}\label{ssec:language}

From \cite{DBLP:conf/fossacs/Glabbeek18}, we can represent a language $\mathscr{L}$ as a pair $(\mathbb{T}, \oper)$, where $\mathbb{T}$ is a set of valid expressions in $\mathscr{L}$, and $\oper$ is a mapping $\oper : \mathbb{T} \to \mathscr{D}$ from $\mathbb{T}$ to a set of meanings $\mathscr{D}$. We also define $A \subseteq \mathbb{T}$, where $A$ is the set of actions.

The expressiveness of two languages, $\mathscr{L}$ and $\mathscr{L}'$ can be measured using a Translation, i.e. a way to map expressions in one language to another
\begin{dfn}[Translation]{dfn:translation}{}
   Via  \cite{DBLP:conf/fossacs/Glabbeek18}, a \textbf{translation} from a language $\mathscr{L}$ to a language $\mathscr{L'}$ is a mapping $\tran : \expres_{\mathscr{L}} \to \expres_{\mathscr{L}'}$
\end{dfn}

The preferred way to measure expressiveness is via relative expressiveness, compared to Absolute expressiveness. \cite{parrowExpressivenessProcessAlgebras2008} Absolute expressiveness measures the way that that processes compare against each other, i.e. if a process in an algebra can be represented by another. Relative expressiveness takes a more robust approach, in trying to encode the operations of the process algebra, and the processes of Absolute Expressiveness can then be constructed via the translated operations. 

\begin{dfn}[Expressiveness]{dfn:expressiveness}{}
   Via  \cite{DBLP:conf/fossacs/Glabbeek18}, a language $\mathscr{L}'$ is \textbf{at least as expressive as $\mathscr{L}$} iff a \textbf{valid} translation from $\mathscr{L}$ into $\mathscr{L}'$ exists.
\end{dfn}

The wording of ``valid'' is intentionally left vague, as there are many notions of validity. Validity is measured using a relation, and the strongest relation there is between two Algebas is a Bisimulation. This is a relation where any behaviour in an algebra can be identically replicated by another, therefore perfectly simulating each other. As we will see, while bisimulation is ideal, it is not always possible to achieve. Listed in \cite{parrowExpressivenessProcessAlgebras2008} is also a range of weaker criterons that are desirable for a translation. One particular criteron that we will focus on is \textbf{compositionality}, which, when achieved, means that operators are valid regardless of the context inside them, which means that any expression will be encodable from one algebra to another by translating smaller and smaller segments of the expression.

\begin{dfn}[Compositionality]{dfn:compositionality}{}
   Via \cite{DBLP:conf/fossacs/Glabbeek18}, a translation $\mathscr{T}$ from $\mathscr{L}$ into a language $\mathscr{L}'$ is \textbf{compositional} if $\trans{X} = X$ for each $X\in \mathscr{X}$, and for each $n$-ary operator $f$ of $\mathscr{L}$ there exists an $n$'ary $\mathscr{L}$-context $C_{f}$ such that $\trans{f(E_{1},\dots,E_{n})} = C_{f}[\trans{E_{1},\dots,\trans{E_{n}}}]$ for any $\mathscr{L}$ expressions $E_{1},\dots,E_{n}\in \mathbb{T}_{\mathscr{L}}$
\end{dfn}


\section{\texorpdfstring{$\csp$}{CSP}}\label{ssec:CSP}
As stated above, our proposed grammar of $\csp$ consists of the operations:
\begin{align*}
   P, Q ::= &\mathrm{STOP} \mid \mathrm{div} \mid a\to P \mid P \sqcap Q \mid P \extchoice Q \mid P \triangleleft Q \mid \\
	&P | |_{A} Q \mid P \backslash A \mid f(P) \mid P \triangle Q \mid P \theta_{A} Q \mid 
\end{align*}
%\mu p.P
where the operators are: \textit{inaction}, \textit{divergence}, \textit{action prefixing}, \textit{internal choice}, \textit{external choice}, \textit{sliding choice}, \textit{parallel composition}, \textit{concealment}, \textit{renaming}, \textit{interrupt}, and \textit{throw}. The operators \textit{inaction}

These can also be represented in the following GSOS table

\import{tables/}{gsos-csp.tex}

\section{\texorpdfstring{$\acptf$}{ACP-tf}}

The process algebra ACP
\import{tables/}{gsos-acp.tex}

\begin{align*}
   P, Q ::= a \mid \delta \mid E + F \mid E . F \mid E \merge F \mid E \leftmerge F \mid E | F \mid \restrict(E) \mid \tau_{I}
\end{align*}
where the operators are: \textit{action}, \textit{deadlock}, \textit{alternative composition}, \textit{sequential composition}, \textit{merge}, \textit{left merge}, \textit{communication merge}, \textit{encapsulation}, \textit{abstraction}

A proposed extension of ACP adds a Functional Renaming operator, as shown in [ON THE EXPRESSIVENESS OF ACP]. From this point forth, we will be using this extension, written as $ACP^{\tau}_{F}$. 

\end{document}
