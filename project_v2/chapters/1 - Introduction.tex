\documentclass[../hons_project.tex]{subfiles}
\begin{document}

With the growing complexities of software and systems of the world, it is key to have methods of modelling more complex systems to get a better understanding of the underlying behaviour behind processes. Efforts have been made in sequential programming as early as the 1930's with Turing Machines and the $\lambda$-calculus. Systems in real life are rarely sequential, however, and usually involve multiple processes acting simultaneously, sometimes even synchronizing to interact with each other to perform tasks. These tasks that involve modelling multiple processes at once are referred to as a \textit{ concurrent system}. 

It is clear to see that brute forcing solutions to these problems are significantly harder than a sequential system - the processing time will grow exponentially as the number of processes increase, and modelling a system like a colony of ants is near impossible. Therefore, we will need some way to formalise these concurrent systems. One category of models for concurrent systems are referred to as ``Process Algebras'', which will be the focus of this paper. These are languages similar to the Algebras of Mathematics, with processes built upon axioms, and often with familiar faces such as addition or multiplication but in slightly different contexts. This way of creation lets us compute results of complex models without the need for equally complex and multidimensional diagrams.

One of the big problems with Process Algebras is the ease of creation of new Algebras. The range of operators can vary language to language, with algebras existing for highly specific use-cases, since, why would you need to include an operator if you never use it? This leads to many different Process Algebras existing, and even multiple variants of a singular Process Algebra with slightly different tweaks added to it. To categorise all these different Algebras leads us to Expressiveness. The end goal of expressiveness is to create a hierarchy of different Process Algebras to see which algebras are more powerful. Put simply, if one Algebra can perform all tasks that another one can do, but not vice-versa, then it is clear that the first Algebra is more expressive.

This hierarchy of Expressiveness is vast and almost impossible to categorise into one paper, so I will be focusing on the expressiveness of two influencial algebras in the history of Concurrency - $\acp$ and $\csp$. In \textbf{Chapter 2}, I will provide background information, a demonstration of a simple process algebra, and an introduction to our two algebras that we will focus on. A more formal definition of $\acp$ and $\csp$ will be defined in \textbf{Chapter 3}, as well as what it means for a language to be ``more expressive'' than another, and the method we will use to achieve it. In \textbf{Chapter 4}, I will define the actual translation between $\csp$ and $\acp$, together with any relevant assistance that will be needed to help define it. In \textbf{Chapter 5}, I will provide a justification for the translation that I defined in the previous section, and finally in \textbf{Chapter 6}, I provide a summary of the results and conclusions of the paper.

\end{document}
