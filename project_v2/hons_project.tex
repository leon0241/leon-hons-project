\documentclass[logo,bsc,singlespacing,parskip,online]{infthesis}
\usepackage{rss/ugcheck}

\usepackage{rss/preamble}
% \overfullrule=20mm


\begin{document}
\begin{preliminary}

\title{A translation from CSP to ACP}

\author{Leon Lee}
\course{Computer Science and Mathematics}
\project{4th Year Project Report}
\date{\today}

\abstract{
   A popular technique for representing Concurrent Systems is categorised into models called Process Algebras. Many Process Algebras exist, and efforts have been made to contextualise different Algebra to each other to find out notions of which Algebra are ``better''.

   In this paper, we use the notion of ``expressiveness'', namely, ``Can one Algebra express more tasks than another'', and I present a translation between two popular Algebras, $\acp$ and $\csp$, that is valid up to a Rooted Branching Bisimulation equivalence.
}

\maketitle

\newenvironment{ethics}
   {\begin{frontenv}{Research Ethics Approval}{\LARGE}}
   {\end{frontenv}\newpage}

\begin{ethics}
This project was planned in accordance with the Informatics Research
Ethics policy. It did not involve any aspects that required approval
from the Informatics Research Ethics committee.

\standarddeclaration
\end{ethics}


\begin{acknowledgements}
Any acknowledgements go here.
\end{acknowledgements}

% \tableofcontents
\end{preliminary}


\chapter{Introduction}
\subfile{chapters/1 - Introduction}

\chapter{Background}
\subfile{chapters/2 - Background}

\chapter{A formal definition of \texorpdfstring{$\csp$}{CSP} and \texorpdfstring{$\acptf$}{ACP}}
\subfile{chapters/3 - Languages.tex}

\chapter{A Translation of \texorpdfstring{$\csp$}{CSP} to \texorpdfstring{$\acptf$}{ACP}}
\subfile{chapters/4 - Translation.tex}


\newpage
\chapter{Validity of the Encoding}
\subfile{chapters/5 - Validity.tex}

\chapter{Conclusion}
\subfile{chapters/6 - Results.tex}

\bibliographystyle{dinat}
\bibliography{project}


\appendix

\chapter{Diagrams}
\subfile{chapters/A1 - Diagrams.tex}

\chapter{Extended Definitions}
\subfile{chapters/A2 - Syntax.tex}

\end{document}
