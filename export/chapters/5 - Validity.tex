%! TEX root = ../hons_project.tex

\documentclass[../hons_project.tex]{subfiles}
\begin{document}

Referring back to our translation of External choice shown in \cref{ssec:external-choice}:
\begin{equation}\tag{\ref{trans:extchoice}}
	\partial_{H_{0}}(\fdef{post}[\Gamma(P)\merge\mathtt{choose}\merge\Gamma(Q)])
\end{equation}
The translation has identical behaviour to $P + Q$ when translating processes with only external actions. However, on processes with $\tau$ actions, the translation is not so trivial. The addition of an internal action in the grammar of $\acptf$ works in our translation's favour for actions such as deferring an $\mathtt{ini}$ tag to the first visible action (see \cref{fig:triggering-tau}). However, our translation largely relies on removing unwanted left-merges and communications through restriction operators. As internal actions are not able to interact with other operators, this results in restrictions that \textit{should} remove all unneccessary actions ending up with unwanted left-over $\tau$ moves.
\newline For example, if we take the process $a \extchoice \tau.b\in \csp$, the resulting translation in $\acptf$ is
\vspace{-5pt}
\begin{equation}
	\partial_{H_{0}}\Bigl(\fdef{post}\Bigl[\Gamma(a) \merge \mathtt{choose} \merge \Gamma(\tau.b)\Bigr]\Bigr) = \partial_{H_{0}}\Bigl(\fdef{post}\Bigl[a_{\mathtt{ini}} \merge \mathtt{choose} \merge \tau.b_{\mathtt{ini}}\Bigr]\Bigr),
\end{equation}
which has the following process graph:

\begin{figure}[!hb!]
	\centering
	% https://q.uiver.app/#q=WzAsMTIsWzAsMywiXFxidWxsZXQiXSxbMiwyLCJcXGJ1bGxldCJdLFsyLDUsIlxcYnVsbGV0Il0sWzQsMSwiXFxidWxsZXQiXSxbNCw0LCJcXGJ1bGxldCJdLFs2LDMsIlxcYnVsbGV0Il0sWzAsMiwiXFxidWxsZXQiXSxbMiwxLCJcXGJ1bGxldCJdLFs0LDAsIlxcYnVsbGV0Il0sWzYsMiwiXFxidWxsZXQiXSxbNCwzLCJcXGJ1bGxldCJdLFsyLDQsIlxcYnVsbGV0Il0sWzAsMSwiXFx0YXUiLDFdLFswLDIsImFfe1xcbWF0aHR0e2luaX19IiwxLHsiY29sb3VyIjpbMjQwLDYwLDc2XSwic3R5bGUiOnsiYm9keSI6eyJuYW1lIjoiZGFzaGVkIn19fSxbMjQwLDYwLDc2LDFdXSxbMSwzLCJiX3tcXG1hdGh0dHtpbml9fSIsMSx7ImNvbG91ciI6WzAsNjAsNzFdLCJzdHlsZSI6eyJib2R5Ijp7Im5hbWUiOiJkYXNoZWQifX19LFswLDYwLDcxLDFdXSxbMSw0LCJhX3tcXG1hdGh0dHtpbml9fSIsMSx7ImNvbG91ciI6WzI0MCw2MCw3Nl0sInN0eWxlIjp7ImJvZHkiOnsibmFtZSI6ImRhc2hlZCJ9fX0sWzI0MCw2MCw3NiwxXV0sWzMsNSwiYV97XFxtYXRodHR7aW5pfX0iLDEseyJjb2xvdXIiOlsyNDAsNjAsNzZdLCJzdHlsZSI6eyJib2R5Ijp7Im5hbWUiOiJkYXNoZWQifX19LFsyNDAsNjAsNzYsMV1dLFsyLDQsIlxcdGF1IiwyLHsiY29sb3VyIjpbMCwwLDQ2XX0sWzAsMCw0NiwxXV0sWzQsNSwiYl97XFxtYXRodHR7aW5pfX0iLDEseyJjb2xvdXIiOlswLDYwLDcxXSwic3R5bGUiOnsiYm9keSI6eyJuYW1lIjoiZGFzaGVkIn19fSxbMCw2MCw3MSwxXV0sWzAsNiwiXFxtYXRodHR7Y2hvb3NlfSIsMCx7ImNvbG91ciI6WzAsMCw0MF0sInN0eWxlIjp7ImJvZHkiOnsibmFtZSI6ImRhc2hlZCJ9fX0sWzAsMCw0MCwxXV0sWzEsNywiXFxtYXRodHR7Y2hvb3NlfSIsMCx7ImNvbG91ciI6WzAsMCw0MF0sInN0eWxlIjp7ImJvZHkiOnsibmFtZSI6ImRhc2hlZCJ9fX0sWzAsMCw0MCwxXV0sWzMsOCwiXFxtYXRodHR7Y2hvb3NlfSIsMix7ImNvbG91ciI6WzAsMCw0MF0sInN0eWxlIjp7ImJvZHkiOnsibmFtZSI6ImRhc2hlZCJ9fX0sWzAsMCw0MCwxXV0sWzUsOSwiXFxtYXRodHR7Y2hvb3NlfSIsMix7ImNvbG91ciI6WzAsMCw0MF0sInN0eWxlIjp7ImJvZHkiOnsibmFtZSI6ImRhc2hlZCJ9fX0sWzAsMCw0MCwxXV0sWzQsMTAsIlxcbWF0aHR0e2Nob29zZX0iLDIseyJjb2xvdXIiOlswLDAsNDBdLCJzdHlsZSI6eyJib2R5Ijp7Im5hbWUiOiJkYXNoZWQifX19LFswLDAsNDAsMV1dLFsyLDExLCJcXG1hdGh0dHtjaG9vc2V9IiwyLHsiY29sb3VyIjpbMCwwLDQwXSwic3R5bGUiOnsiYm9keSI6eyJuYW1lIjoiZGFzaGVkIn19fSxbMCwwLDQwLDFdXSxbMTEsMTAsIlxcdGF1IiwxXSxbMTAsOSwiYl97XFxtYXRodHR7aW5pfX0iLDEseyJjb2xvdXIiOlswLDYwLDcxXSwic3R5bGUiOnsiYm9keSI6eyJuYW1lIjoiZGFzaGVkIn19fSxbMCw2MCw3MSwxXV0sWzcsOCwiYl97XFxtYXRodHR7aW5pfX0iLDEseyJjb2xvdXIiOlswLDYwLDcxXSwic3R5bGUiOnsiYm9keSI6eyJuYW1lIjoiZGFzaGVkIn19fSxbMCw2MCw3MSwxXV0sWzYsNywiXFx0YXUiLDEseyJjb2xvdXIiOlswLDAsNDZdfSxbMCwwLDQ2LDFdXSxbNiwxMSwiYV97XFxtYXRodHR7aW5pfX0iLDEseyJjb2xvdXIiOlsyNDAsNjAsNzZdLCJzdHlsZSI6eyJib2R5Ijp7Im5hbWUiOiJkYXNoZWQifX19LFsyNDAsNjAsNzYsMV1dLFs3LDEwLCJhX3tcXG1hdGh0dHtpbml9fSIsMSx7ImNvbG91ciI6WzI0MCw2MCw3Nl0sInN0eWxlIjp7ImJvZHkiOnsibmFtZSI6ImRhc2hlZCJ9fX0sWzI0MCw2MCw3NiwxXV0sWzgsOSwiYV97XFxtYXRodHR7aW5pfX0iLDEseyJjb2xvdXIiOlsyNDAsNjAsNzZdLCJzdHlsZSI6eyJib2R5Ijp7Im5hbWUiOiJkYXNoZWQifX19LFsyNDAsNjAsNzYsMV1dLFswLDExLCJhIiwxLHsiY29sb3VyIjpbMjQwLDYwLDYwXX0sWzI0MCw2MCw2MCwxXV0sWzEsMTAsImEiLDEseyJjb2xvdXIiOlsyNDAsNjAsNjBdfSxbMjQwLDYwLDYwLDFdXSxbMyw5LCJhIiwxLHsiY29sb3VyIjpbMjQwLDYwLDc5XX0sWzI0MCw2MCw3OSwxXV0sWzEsOCwiYiIsMSx7ImNvbG91ciI6WzAsNjAsNjBdfSxbMCw2MCw2MCwxXV1d
	\[\begin{tikzcd}[cramped, row sep=scriptsize]
			&&&& \bullet \\
			&& \bullet && \bullet \\
			\bullet && \bullet &&&& \bullet \\
			\bullet &&&& \bullet && \bullet \\
			&& \bullet && \bullet \\
			&& \bullet
			\arrow["{a_{\mathtt{ini}}}"{description}, color={rgb,255:red,157;green,157;blue,231}, dashed, from=1-5, to=3-7]
			\arrow["{b_{\mathtt{ini}}}"{description}, color={rgb,255:red,225;green,137;blue,137}, dashed, from=2-3, to=1-5]
			\arrow["{a_{\mathtt{ini}}}"{description}, color={rgb,255:red,157;green,157;blue,231}, dashed, from=2-3, to=4-5]
			\arrow["{\mathtt{choose}}"', color={rgb,255:red,102;green,102;blue,102}, dashed, from=2-5, to=1-5]
			\arrow["a"{description}, color={rgb,255:red,169;green,169;blue,234}, from=2-5, to=3-7]
			\arrow["{a_{\mathtt{ini}}}"{description}, color={rgb,255:red,157;green,157;blue,231}, dashed, from=2-5, to=4-7]
			\arrow["\tau"{description}, color={rgb,255:red,117;green,117;blue,117}, from=3-1, to=2-3]
			\arrow["{a_{\mathtt{ini}}}"{description}, color={rgb,255:red,157;green,157;blue,231}, dashed, from=3-1, to=5-3]
			\arrow["b"{description}, color={rgb,255:red,214;green,92;blue,92}, from=3-3, to=1-5]
			\arrow["{\mathtt{choose}}", color={rgb,255:red,102;green,102;blue,102}, dashed, from=3-3, to=2-3]
			\arrow["{b_{\mathtt{ini}}}"{description}, color={rgb,255:red,225;green,137;blue,137}, dashed, from=3-3, to=2-5]
			\arrow["a"{description}, color={rgb,255:red,92;green,92;blue,214}, from=3-3, to=4-5]
			\arrow["{a_{\mathtt{ini}}}"{description}, color={rgb,255:red,157;green,157;blue,231}, dashed, from=3-3, to=5-5]
			\arrow["{\mathtt{choose}}", color={rgb,255:red,102;green,102;blue,102}, dashed, from=4-1, to=3-1]
			\arrow["\tau"{description}, from=4-1, to=3-3]
			\arrow["a"{description}, color={rgb,255:red,92;green,92;blue,214}, from=4-1, to=5-3]
			\arrow["{a_{\mathtt{ini}}}"{description}, color={rgb,255:red,157;green,157;blue,231}, dashed, from=4-1, to=6-3]
			\arrow["{b_{\mathtt{ini}}}"{description}, color={rgb,255:red,225;green,137;blue,137}, dashed, from=4-5, to=3-7]
			\arrow["{\mathtt{choose}}"', color={rgb,255:red,102;green,102;blue,102}, dashed, from=4-7, to=3-7]
			\arrow["\tau"{description}, from=5-3, to=4-5]
			\arrow["{\mathtt{choose}}"', color={rgb,255:red,102;green,102;blue,102}, dashed, from=5-5, to=4-5]
			\arrow["{b_{\mathtt{ini}}}"{description}, color={rgb,255:red,225;green,137;blue,137}, dashed, from=5-5, to=4-7]
			\arrow["{\mathtt{choose}}"', color={rgb,255:red,102;green,102;blue,102}, dashed, from=6-3, to=5-3]
			\arrow["\tau"', color={rgb,255:red,117;green,117;blue,117}, from=6-3, to=5-5]
		\end{tikzcd}\]
	\caption{Counterexample for strong bisimilarity for the processes $P \extchoice \tau.b$. The result of the translation is $a.\tau + \tau.(a+b) \not\leftrightarroweq a + \tau.(a+b)$}
	\label{fig:sb-counterexample}
\end{figure}

\newpage
\section{Prerequisites and Helper Theorems}
\subsection{Rooted Branching bisimilarity}

Recall from Definition \ref{dfn:rooted-branching} the definition of branching bisimilarity and rooted branching bisimilarity:

\begin{rcl}[Branching Bisimulation]{rcl:rbb}{3.10}
	A relation $R$ is a \textbf{Branching bisimulation}\footnote{\label{foot:successful-term}In this definition, the rules for Successful termination are omitted. The full rule set is defined in Definition \ref{dfn:rooted-branching}} between $P$ and $Q$ if:
	\begin{enumerate}
		\item The roots of $P$ and $Q$ are related by $R$.
		\item If $s \prightarrow{\alpha} s'$ for $\alpha\in \Sigma_{\tau}$ is an edge in $P$, and $s R t$, then either:
		      \begin{enumerate}[label=\alph*)]
			      \item $\alpha= \tau$ and $s' R t$.
			      \item $\exists t \Rightarrow t_{1} \prightarrow{\alpha} t'$ such that $s R t_{1}$ and $s R t'$.\footnote{Here, we use the symbol $\Rightarrow$ to mean a chain of $\tau$ actions, possibly none.}
		      \end{enumerate}
		\item[4.] If $t \prightarrow{\alpha} t'$ for $\alpha\in \Sigma_{\tau}$ is an edge in $Q$, and $s R t$, then either:
		      \begin{enumerate}[label=\alph*)]
			      \item $\alpha= \tau$ and $s R t'$.
			      \item $\exists s \Rightarrow s_{1} \prightarrow{\alpha} s'$ such that $s_{1} R t$ and $s' R t$.
		      \end{enumerate}
	\end{enumerate}

	$R$ is called a \textbf{Rooted branching bisimulation}\footref{foot:successful-term} if the following root condition is also satisfied:
	\begin{itemize}
		\item If $\mathrm{root}(P) \prightarrow{\alpha} s'$ for $\alpha\in \Sigma_{\tau}$, then there is a $t'$ with $\mathrm{root}(Q)\prightarrow{\alpha} t'$ and $s' R t'$.
		\item If $\mathrm{root}(Q) \prightarrow{\alpha} t'$ for $\alpha\in \Sigma_{\tau}$, then there is an $s'$ with $\mathrm{root}(P)\prightarrow{\alpha} s'$ and $s' R t'$.
	\end{itemize}
\end{rcl}

\subsection{Lemmas and Theorems}

We will now state some Lemmas to aid our proof of the validity of our translation.

\begin{thm}[Expansion of \texorpdfstring{$\acp$}{ACP} merge]{lma:acp-merge-simplification}{}
	The process
	\begin{equation}
		\partial_{H_{0}}(P \merge Q),
	\end{equation}
	where $P, Q\in \acptf$, $P$ is of the form $\alpha.P'$, $Q$ is of the form $\beta.Q'$\footnote{Here, both $\alpha$ and $\beta$ are actions in the set $\Sigma \cup \{\tau\}$}, and $H_{0}$ is the set of working space operators as defined in Definition \ref{dfn:sets}, can be expanded to the following process:
	\begin{align}
		\begin{split}
			\partial_{H_{0}}(P \merge Q) = & \partial_{H_{0}}(\alpha).\partial_{H_{0}}(P' \merge Q))                                                                       \\
			                               & + \partial_{H_{0}}(\beta).\partial_{H_{0}}(Q' \merge P)) + \partial_{H_{0}}(\alpha \mid \beta).\partial_{H_{0}}(P' \merge Q')
		\end{split}
	\end{align}
\end{thm}

\begin{proof}
	We can derive the following equation from the axioms of $\acp$ shown in \ref{table:acpt-axioms}, starting by expanding the Axiom CM1.
	\begin{align*}
		\partial_{H_{0}}( P \merge Q ) & = \partial_{H_{0}}\bigl( P \leftmerge Q + Q \leftmerge P + P \mid Q\bigr)                                                                                                               \\
		                               & = \partial_{H_{0}}\bigl( \alpha.P' \leftmerge Q + \beta.Q' \leftmerge P + \alpha.P' \mid \beta.Q'\bigr)                                                                                 \\
		{\color{blue}\text{CM3, CM7}}  & = \partial_{H_{0}}\bigl( \alpha.(P' \merge Q) + \beta.(Q' \merge P) + (\alpha \mid \beta).(P' \merge Q')\bigr)                                                                          \\
		{\color{blue}\text{D3}}        & = \partial_{H_{0}}(\alpha.(P' \merge Q)) + \partial_{H_{0}}(\beta.(Q' \merge P)) + \partial_{H_{0}}\bigl((\alpha \mid \beta).(P' \merge Q')\bigr)                                       \\
		{\color{blue}\text{D4}}        & = \partial_{H_{0}}(\alpha).\partial_{H_{0}}(P' \merge Q)) + \partial_{H_{0}}(\beta).\partial_{H_{0}}(Q' \merge P)) + \partial_{H_{0}}(\alpha \mid \beta).\partial_{H_{0}}(P' \merge Q') \\
	\end{align*}
\end{proof}

\begin{lma}[]{lma:merge-simplification-a-h}{}
	For processes $P,\,Q\in \acptf$, if $P$ is of the form $\alpha.P'$ where $\alpha\in \Sigma_{0} \cup \{\tau\}$, and $Q$ is of the form $\beta.Q'$ where $\beta\in H_{0}$ (where $H_{0}$ is the set of working space operators as defined in Definition \ref{dfn:sets}), and such that $\alpha \mid \beta$ is not defined on $\mid$, then
	\begin{equation}
		\partial_{H_{0}}(P \merge Q) = \alpha.\partial_{H_{0}}(P' \merge Q).
	\end{equation}
\end{lma}

\begin{proof}
	Starting from Theorem \ref{lma:acp-merge-simplification}, we can derive the following equation:
	\begin{align*}
		\partial_{H_{0}}(P \merge Q) & = \partial_{H_{0}}(\alpha).\partial_{H_{0}}(P' \merge Q)) + \partial_{H_{0}}(\beta).\partial_{H_{0}}(Q' \merge P)) + \partial_{H_{0}}(\alpha \mid \beta).\partial_{H_{0}}(P' \merge Q') \\
		                             & = \alpha.\partial_{H_{0}}(P' \merge Q) + \delta.\partial_{H_{0}}(Q' \merge P) + \partial_{H_{0}}(\delta).\partial_{H_{0}}(P' \merge Q')                                                 \\
		                             & = \alpha.\partial_{H_{0}}(P' \merge Q) + {\color{red}\delta.\partial_{H_{0}}(Q' \merge P)} + {\color{red}\delta.\partial_{H_{0}}(P' \merge Q')}                                         \\
		                             & = \alpha.\partial_{H_{0}}(P' \merge Q)
	\end{align*}
	%
	% This can be recursively expanded, the next level is shown below (assuming the next action in $P$ is named $\gamma$):
	% \[\alpha.\partial_{H_{0}}(P' \merge Q) = \alpha.\partial_{H_{0}}\bigl(\gamma. (P'' \merge Q) + {\color{red}\phi.(Q' \merge P)} + {\color{red}\phi.(P' \merge Q')}\bigr)\]
	% Therefore, it is clear that after the entire process has been expanded, the equation will result in the process
	% \begin{equation}
	% 	\partial_{H_{0}}({\alpha.P'}).
	% \end{equation}
\end{proof}

\begin{lma}[]{lma:merge-simplification-a-a}{}
	For processes $P,\,Q\in \acptf$, if $P$ is of the form $\alpha.P'$ and $Q$ is of the form $\beta.Q'$, where $\alpha,\,\beta\in \Sigma_{0} \cup \{\tau\}$, and such that $\alpha \mid \beta$ is not defined on $\mid$, then
	\begin{equation}
		\partial_{H_{0}}(P \merge Q) = \alpha.\partial_{H_{0}}(P' \merge Q) + \beta.\partial_{H_{0}}(Q' \merge P).
	\end{equation}
\end{lma}

\begin{proof}
	Starting from Theorem \ref{lma:acp-merge-simplification}, we can derive the following equation:
	\begin{align*}
		\partial_{H_{0}}(P \merge Q) & = \partial_{H_{0}}(\alpha).\partial_{H_{0}}(P' \merge Q) + \partial_{H_{0}}(\beta).\partial_{H_{0}}(Q' \merge P) + \partial_{H_{0}}(\alpha \mid \beta).\partial_{H_{0}}(P' \merge Q') \\
		                             & = \alpha.\partial_{H_{0}}(P' \merge Q) + \beta.\partial_{H_{0}}(Q' \merge P) + \partial_{H_{0}}(\delta).\partial_{H_{0}}(P' \merge Q')                                                \\
		                             & = \alpha.\partial_{H_{0}}(P' \merge Q) + \beta.\partial_{H_{0}}(Q' \merge P) + {\color{red} \delta.\partial_{H_{0}}(P' \merge Q') }                                                   \\
		                             & = \alpha.\partial_{H_{0}}(P' \merge Q) + \beta.\partial_{H_{0}}(Q' \merge P)
	\end{align*}
\end{proof}

\begin{lma}[]{lma:merge-simplification-h-h}{}
	For processes $P,\,Q\in \acptf$, if $P$ is of the form $\alpha.P'$ and $Q$ is of the form $\beta.Q'$, where $\alpha,\,\beta\in H_{0}$ (where $H_{0}$ is the set of working space operators as defined in Definition \ref{dfn:sets}), and such that $\alpha \mid \beta = \phi$, then
	\begin{equation}
		\partial_{H_{0}}(P \merge Q) = \phi.\partial_{H_{0}}(P' \merge Q').
	\end{equation}
\end{lma}

\begin{proof}
	Starting from Theorem \ref{lma:acp-merge-simplification}, we can derive the following equation:
	\begin{align*}
		\partial_{H_{0}}(P \merge Q) & = \partial_{H_{0}}(\alpha).\partial_{H_{0}}(P' \merge Q) + \partial_{H_{0}}(\beta).\partial_{H_{0}}(Q' \merge P) + \partial_{H_{0}}(\alpha \mid \beta).\partial_{H_{0}}(P' \merge Q') \\
		                             & = \delta.\partial_{H_{0}}(P' \merge Q) + \delta.\partial_{H_{0}}(Q' \merge P) + \partial_{H_{0}}(\phi).\partial_{H_{0}}(P' \merge Q')                                                 \\
		                             & = {\color{red}\delta.\partial_{H_{0}}(P' \merge Q)} + {\color{red}\delta.\partial_{H_{0}}(Q' \merge P)} + \phi.\partial_{H_{0}}(P' \merge Q')                                         \\
		                             & = \phi.\partial_{H_{0}}(P' \merge Q')
	\end{align*}
\end{proof}

\begin{lma}[]{lma:stagnant-inis}{}
	For processes $P, Q\in \acptf$, if $P$ is of the form $\alpha.P$, and $Q$ is of the form $\taus.b.Q$ where $b\in H_{0}$ (where $H_{0}$ is the set of working space operators as defined in Definition \ref{dfn:sets}), $\alpha$ does not communicate with $b$, and $\taus$ indicates a chain of consecutive $\tau$, possibly $0$, then
	\begin{equation}
		\partial_{H_{0}} (a.P \merge \taus.b.Q) = \alpha.\partial_{H_{0}}(P) \merge \taus.
	\end{equation}
\end{lma}

\begin{proof}
	Starting from Lemma \ref{lma:merge-simplification-a-a}, we can derive the following equation:
	\begin{align*}
		\partial_{H_{0}}(P \merge Q) & = \alpha.\partial_{H_{0}}((P' \merge Q)) + \beta.\partial_{H_{0}}((Q' \merge P)) \\
		\partial_{H_{0}}(P \merge Q) & = \alpha.\partial_{H_{0}}((P' \merge Q)) + \tau.\partial_{H_{0}}((Q' \merge P))
	\end{align*}
	Note that for $n$ number of consecutive $\tau$ moves in $Q$, $Q'$ has $n - 1$ consecutive $\tau$ moves. We can recursively apply Lemma \ref{lma:merge-simplification-a-a} on both sides of the processes, splitting them up into processes with progressively smaller and smaller number of $\tau$ moves, until we reach a state where we can apply Lemma \ref{lma:merge-simplification-a-h}, at which point we have the process
	\begin{align*}
		\partial_{H_{0}}(P \merge Q) & = \underbrace{\tau.\tau. \cdots. \tau}_{n}.\bigl(\partial_{H_{0}}(\beta). \cdots + \alpha.\partial_{H_{0}}(P')\bigr) + \underbrace{\tau.\tau. \cdot\cdot. \tau}_{n-1}.\bigl(\partial_{H_{0}}(\beta). \cdots + \alpha.\partial_{H_{0}}(P')\bigr) \\
		                             & + \cdots + \tau.\bigl(\partial_{H_{0}}(\beta). \cdots + \alpha.\partial_{H_{0}}(P')\bigr)  + \bigl(\partial_{H_{0}}(\beta). \cdots + \alpha.\partial_{H_{0}}(P')\bigr)                                                                          \\
	\end{align*}
	Via Lemma \ref{lma:merge-simplification-a-h}, this can then be simplified to
	\begin{align*}
		\partial_{H_{0}}(P \merge Q) & = \underbrace{\tau.\tau. \cdots. \tau}_{n}.\bigl(\alpha.\partial_{H_{0}}(P')\bigr) + \underbrace{\tau.\tau. \cdots. \tau}_{n-1}.\bigl(\alpha.\partial_{H_{0}}(P')\bigr) \\
		                             & + \cdots + \tau.\bigl(\alpha.\partial_{H_{0}}(P')\bigr)  + \bigl(\alpha.\partial_{H_{0}}(P')\bigr)                                                                      \\
	\end{align*}
	Since $\tau$ actions cannot communicate, this process is equivalent to
	\[\partial_{H_{0}} (a.P \merge \taus.b.Q) = \alpha.\partial_{H_{0}}(P) \merge \taus.\]

\end{proof}


\begin{lma}[]{lma:tau-on-p-process}{}
	For a process $P\in \csp$ where $\trans{P}\in\acptf$ is strongly bisimilar to $P$, a process $\taus\merge\trans{P}$ is Branching bisimilar to the process $P$. Here, I use the notation of $\taus$ indicating a chain of $\tau$, possibly $0$.
\end{lma}

\begin{proof}
	We have the rule that
	\[\tau \merge P \rbrb P\]
	Now, simply take $P = \tau.P$ and inductively, this law will work for arbitrary $\tau$ actions.
\end{proof}

We can now work towards a proof that our translation is valid up to Rooted Branching bisimilarity.


\newpage
\section{Proof of Rooted Branching bisimilarity}\label{sc:rbb-proof}

We define a rooted branching bisimulation relation.
\begin{dfn}[Rooted branching bisimulation Relation]{dfn:bisim-relation}{}
	Let $\tcsp$ be the expressions in the language $\csp$, and $\tacp$ be expressions in the language $\acptf$. We use the translation $\tran : \tcsp \to \tacp$ as defined in \ref{dfn:trans}.

	We now define a Rooted branching bisimulation between $\tcsp$ and $\tacp$:
	\[
		\rbrb\, := \{(P, \trans{P}) \mid P\in \csp\}
	\]
\end{dfn}

\subsection{Parallel Composition}\label{ssec:proof-pcomp}

The \textbf{Parallel Composition} operator $\pcomp$ is defined as the following equation:
\[\trans{P \pcomp Q}     = \partial_{H_{0}}\Bigl(\fdef{post}\Bigl[\fdef{syn}(\trans{P}) \merge \fdef{syn}(\trans{Q})\Bigr]\Bigr)\]

Let $P, Q \in \csp$ be two processes. We want to show that $P \pcomp Q \rbrb \trans{P \pcomp Q}$. i.e.: we want to show that any move in the original process can be replicable in the translated process up to rooted branching bisimulation. The different moves a process can take with the operator $\pcomp$ are defined in the SOS semantics shown in \cref{table:CSP}, and the relevant rules are also shown in \ref{sos:pcomp}. By exhausting all possible moves that $P \pcomp Q$ can take, and confirming that $\trans{P \pcomp Q}$ can also take them, up to rooted branching bisimilarity, we will have shown that the equivalence holds true.

\begin{itemize}[leftmargin=*]
	\item Let $P'$ such that $P \prightarrow{\alpha} P'$, and $\alpha\not\in A$ for the target set $A$. $\fdef{syn}$ will therefore not affect $\alpha$. In the domain of CSP, this results in the process:
	      \[P \pcomp Q \prightarrow{a} P' \pcomp Q\]
	      We will show that the translation can also take this move. An action $\beta$ in $Q$ will be such that either $\beta\in H_{0}$, in which case we can apply Lemma \ref{lma:merge-simplification-a-h},
	      \begin{align*}
		      \partial_{H_{0}}\Bigl(\fdef{post}\Bigl[\fdef{syn}(\trans{P}) \merge & \fdef{syn}(\trans{Q})\Bigr]\Bigr)                                                                                           \\
		                                                                          & = \alpha.\partial_{H_{0}}\Bigl(\fdef{post}\Bigl[\fdef{syn}(\trans{P'}) \merge \fdef{syn}(\trans{Q})\Bigr]\Bigr)             \\
		                                                                          & \prightarrow{\alpha} \partial_{H_{0}}\Bigl(\fdef{post}\Bigl[\fdef{syn}(\trans{P'}) \merge \fdef{syn}(\trans{Q})\Bigr]\Bigr)
	      \end{align*}
	      or $\beta$ will be in $\Sigma_{0} \cup \tau$, in which case we can apply Lemma \ref{lma:merge-simplification-a-a},
	      \begin{align*}
		      \partial_{H_{0}}\Bigl(\fdef{post}\Bigl[\fdef{syn}(\trans{P}) \merge & \fdef{syn}(\trans{Q})\Bigr]\Bigr)                                                                                           \\
		                                                                          & = \alpha.\partial_{H_{0}}\Bigl(\fdef{post}\Bigl[\fdef{syn}(\trans{P'}) \merge \fdef{syn}(\trans{Q})\Bigr]\Bigr)             \\
		                                                                          & +\beta.\partial_{H_{0}}\Bigl(\fdef{post}\Bigl[\fdef{syn}(\trans{Q'}) \merge \fdef{syn}(\trans{P})\Bigr]\Bigr)               \\
		                                                                          & \prightarrow{\alpha} \partial_{H_{0}}\Bigl(\fdef{post}\Bigl[\fdef{syn}(\trans{P'}) \merge \fdef{syn}(\trans{Q})\Bigr]\Bigr)
	      \end{align*}
	      Note that the $\alpha$ and $\beta$ moves can move out of $\fdef{post}$ because we specified that $a,b\in \Sigma_{0} \cup \tau$. This move is therefore strongly bisimilar to $P' \pcomp Q$, even for internal actions.
	\item Let $Q'$ such that $Q \prightarrow{\alpha} Q'$, and $\alpha\not\in A$ for the target set $A$. This will give the same result as shown above, and is strongly bisimilar.
	\item Let $P'$ such that $P \prightarrow{a} P'$ and $Q'$ s.t. $Q \prightarrow{a} Q'$, where $a\in A$. Since $a_{\mathtt{syn}}\in H_{0}$ and $a_{\mathtt{syn}} \mid a_{\mathtt{syn}} = a_{\mathtt{post}}$, we apply Lemma \ref{lma:merge-simplification-h-h}:
	      \begin{align*}
		      \trans{P \pcomp Q} & =  \partial_{H_{0}}\Bigl(\fdef{post}\Bigl[\fdef{syn}(\trans{P}) \merge \fdef{syn}(\trans{Q})\Bigr]\Bigr)                                                        \\
		                         & =                     \partial_{H_{0}}\Bigl(\fdef{post}\Bigl[a_{\mathtt{syn}}.\fdef{syn}(\trans{P'}) \merge a_{\mathtt{syn}}.\fdef{syn}(\trans{Q'})\Bigr]\Bigr) \\
		                         & \prightarrow{a}  \partial_{H_{0}}\Bigl(\fdef{post}\Bigl[\fdef{syn}(\trans{P'}) \merge \fdef{syn}(\trans{Q'})\Bigr]\Bigr)
	      \end{align*}
	      Note that the move $a$ results from $\fdef{post}$ applied to $a_{\mathtt{post}}$, the result of the communication. This process is strongly bisimilar to $P' \pcomp Q'$.
\end{itemize}
Therefore, our translation for Parallel Composition is valid up to strong bisimilarity.

\subsection{External choice}\label{ssec:proof-extchoice}

From \cref{ssec:external-choice}, our translation of External Choice is:
\begin{equation}\tag{\ref{trans:extchoice}}
	\trans{P \extchoice Q} = \partial_{H_{0}}\Bigl(\fdef{post}\Bigl[\Gamma[\trans{P}]\, \merge \,\mathtt{choose}\, \merge \,\Gamma[\trans{Q}]\Bigl]\Bigr)
\end{equation}
\begin{proof}

	Let $P, Q \in \csp$ be two processes. We proceed similarly to the proof of Parallel Composition, by looking at the SOS rules laid out in \ref{table:CSP}, or in \ref{sos:extchoice}, and exhausting all possible moves that $P \extchoice Q$ can take, confirming that $\trans{P \extchoice Q}$ can also take them up to rooted branching bisimilarity.

	\begin{itemize}[leftmargin=*]
		\item Let $P'$ such that $P \prightarrow{a} P'$. In the domain of CSP, this results in the process:
		      \begin{equation}
			      P \square Q \prightarrow{a} P'
		      \end{equation}
		      Now working in $\acptf$, we want to show the translation is valid up to rooted branching bisimulation, i.e. $\trans{P'} \rbrb \trans{P'}$, or in general, $\trans{\mathcal{P}} \rbrb \mathcal{P}$ for any $\mathcal{P}\in \csp$. From the definition of the Triggering operator \ref{oper:triggering}, we can derive the following equation:
		      \begin{equation}\label{eq:triggering-proofthing}
			      \Gamma(\trans{P}) = a_{\mathtt{ini}}.\trans{P'}
		      \end{equation}

		      Since $\mathtt{choose}\in H_{0}$, and $a_{\mathtt{ini}} \mid \mathtt{choose} = a_{\mathtt{post}}$, we can derive the following process from Lemma \ref{lma:merge-simplification-h-h}:
		      \begin{align*}
			      \trans{P \extchoice Q} & = \, \partial_{H_{0}}\Bigl(\fdef{post}\Bigl[\Gamma[\trans{P}]\merge \mathtt{choose}\merge \Gamma[\trans{Q}]\Bigl]\Bigr)           \\
			                             & = \, \partial_{H_{0}}\Bigl(\fdef{post}\Bigl[a_{\mathtt{ini}}.\trans{P'}\merge \mathtt{choose}\merge \Gamma[\trans{Q}]\Bigl]\Bigr) \\
			                             & \prightarrow{a}  \,\partial_{H_{0}}\Bigl(\fdef{post}\Bigl[\trans{P'}\merge\Gamma[\trans{Q}]\Bigl]\Bigr)
		      \end{align*}
		      Note that the move $a$ results from $\fdef{post}$ applied to $a_{post}$, similarly to in the parallel composition operator. This is a common occurrence, and from now on we will accept that this is the behaviour that will happen.

		      This is not yet a process that is comparable to $P'$, so we look at the next step. Due to the Triggering operator $\Gamma$ being applied to $Q$, the only communicable action of a trace $Q_{c}$ of $Q$ will be one tagged with an \texttt{ini}, with some number of $\tau$ actions behind it. Via Lemma \ref{lma:stagnant-inis}, any of the actions past $\taus$ will get restricted, leaving:
		      \[
			      \partial_{H_{0}}\Bigl(\fdef{post}\Bigl[\trans{P'}\merge\Gamma[\trans{Q_{c}}]\Bigl]\Bigr) \implies \partial_{H_{0}}\Bigl(\fdef{post}\Bigl[\trans{P'}\merge\taus\Bigl]\Bigr) \implies \trans{P'}\merge\taus
		      \]
		      for every trace $Q_{c}$ in $Q$. Via Lemma \ref{lma:tau-on-p-process}, this process is branching bisimilar to the process $\trans{P'}$. From this, we can see the union of every branch in $Q$ is at coarsest branching bisimilar, and therefore as the first action is related up to strong bisimilarity, taking an external action on $P$ is valid up to rooted branching bisimilarity.

		\item Let $P'$ such that $P \prightarrow{\tau} P'$. In the domain of $\csp$, this results in the process:
		      \begin{equation}
			      P \extchoice Q \prightarrow{\tau} P' \extchoice Q
		      \end{equation}

		      Now working in $\acptf$, we want to show that the translation is valid up to RBB, i.e. $\trans{P'\extchoice Q} \rbrb P'\extchoice Q$. Via Lemma \ref{lma:merge-simplification-a-h}, we can now derive the following equation:
		      \begin{align*}
			       & \partial_{H_{0}}\Bigl(f_{\mathtt{post}}\Bigl[\Gamma[\trans{P}]\, | | \,\mathtt{choose}\, | | \,\Gamma[\trans{Q}]\Bigl]\Bigr) \prightarrow{\tau} \\ &\partial_{H_{0}}\Bigl(f_{\mathtt{post}}\Bigl[\Gamma[\trans{P'}]\, | | \,\mathtt{choose}\, | | \,\Gamma[\trans{Q}]\Bigl]\Bigr)
		      \end{align*}
		      This process is strongly bisimilar to $P' \extchoice Q$, therefore a $\tau$ action on $P$ is also valid up to rooted branching bisimilarity.
		\item The same logic from Item 1 and 2 can be applied in reverse to $Q$ and $P$ to also obtain processes that satisfy rooted branching bisimilarity.
	\end{itemize}
	We have now exhausted all cases, and therefore can conclude that our translation of $\csp$ External Choice is valid up to rooted branching bisimilarity.
\end{proof}

\subsection{Sliding Choice}\label{ssec:proof-sliding}
From \cref{ssec:sliding-choice}, our translation of the \textbf{Sliding Choice} operator $\sliding$ is defined as the following equation:
\begin{equation}\tag{\ref{trans:sliding}}
	\trans{P \sliding Q}   = \tau_{\{\mathtt{shift}\}}\Bigl(\partial_{H_{0}}\Bigl(f_{\mathtt{post}}\Bigl[ \Gamma(\trans{P}) \merge \mathrm{choose} \merge \mathtt{shift}_{\mathtt{ini}} . \trans{Q}) \Bigr]\Bigr)\Bigr)
\end{equation}

\begin{proof}
	We proceed in the same manner as the previous operators.

	\begin{itemize}[leftmargin=*]
		\item Let $P'$ such that $P \prightarrow{a} P'$. Since $\mathtt{choose}\in H_{0}$, and $\mathtt{choose} \mid a_{\mathtt{ini}} = a$, we apply Lemma \ref{lma:merge-simplification-h-h} using \cref{eq:triggering-proofthing}:
		      \begin{align*}
			      \trans{P \sliding Q}   = & \tau_{\{\mathtt{shift}\}}\Bigl(\partial_{H_{0}}\Bigl(f_{\mathtt{post}}\Bigl[ \Gamma(\trans{P}) \merge \mathrm{choose} \merge \mathtt{shift}_{\mathtt{ini}} . \trans{Q}) \Bigr]\Bigr)\Bigr)           \\
			      =                        & \tau_{\{\mathtt{shift}\}}\Bigl(\partial_{H_{0}}\Bigl(f_{\mathtt{post}}\Bigl[ a_{\mathtt{ini}}.\trans{P'} \merge \mathrm{choose} \merge \mathtt{shift}_{\mathtt{ini}} . \trans{Q}) \Bigr]\Bigr)\Bigr) \\
			      \prightarrow{a}          & \tau_{\{\mathtt{shift}\}}\Bigl(\partial_{H_{0}}\Bigl(f_{\mathtt{post}}\Bigl[\trans{P'} \merge \mathtt{shift}_{\mathtt{ini}} . \trans{Q}) \Bigr]\Bigr)\Bigr)
		      \end{align*}
		      Nothing in $\trans{P'}$ can communicate with the action $\mathtt{shift}_{\mathtt{ini}}$, therefore the process is strongly bisimilar to $P'$.
		\item Let $P'$ such that $P \prightarrow{\tau} P'$. Since $\mathtt{choose}\in H_{0}$, and $\tau$ actions cannot communicate, we apply Lemma \ref{lma:merge-simplification-a-h}:
		      \begin{align*}
			      \trans{P \sliding Q}   = & \tau_{\{\mathtt{shift}\}}\Bigl(\partial_{H_{0}}\Bigl(f_{\mathtt{post}}\Bigl[ \Gamma(\trans{P}) \merge \mathrm{choose} \merge \mathtt{shift}_{\mathtt{ini}} . \trans{Q}) \Bigr]\Bigr)\Bigr)       \\
			      =                        & \tau_{\{\mathtt{shift}\}}\Bigl(\partial_{H_{0}}\Bigl(f_{\mathtt{post}}\Bigl[ \tau.\Gamma(\trans{P'}) \merge \mathrm{choose} \merge \mathtt{shift}_{\mathtt{ini}} . \trans{Q}) \Bigr]\Bigr)\Bigr) \\
			      \prightarrow{\tau}       & \tau_{\{\mathtt{shift}\}}\Bigl(\partial_{H_{0}}\Bigl(f_{\mathtt{post}}\Bigl[\Gamma(\trans{P'}) \merge \mathtt{choose} \merge \mathtt{shift}_{\mathtt{ini}} . \trans{Q}) \Bigr]\Bigr)\Bigr)
		      \end{align*}
		      This process is strongly bisimilar to $P' \sliding Q$.
		\item Let $Q'$ such that $Q \prightarrow{\alpha} Q'$. Since $\mathtt{choose}\in H_{0}$, and the first action of the process must be $\mathtt{shift}_{\mathtt{ini}}$, we apply Lemma \ref{lma:merge-simplification-h-h}:
		      \begin{align*}
			      \trans{P \sliding Q}   = & \tau_{\{\mathtt{shift}\}}\Bigl(\partial_{H_{0}}\Bigl(f_{\mathtt{post}}\Bigl[ \Gamma(\trans{P}) \merge \mathrm{choose} \merge \mathtt{shift}_{\mathtt{ini}} . \trans{Q}) \Bigr]\Bigr)\Bigr) \\
			      \prightarrow{\tau}       & \tau_{\{\mathtt{shift}\}}\Bigl(\partial_{H_{0}}\Bigl(f_{\mathtt{post}}\Bigl[\Gamma(\trans{P'}) \merge \mathtt{shift}.\trans{Q}) \Bigr]\Bigr)\Bigr)
		      \end{align*}
		      This process is branching bisimilar to $Q$. The action $\mathtt{shift}$ will be abstracted by the $\tau$, which is a strongly bisimilar action, but the process $\Gamma(\trans{P})$ can still have $\tau$ actions, which via Lemma \cref{lma:stagnant-inis} will cause the process to derive to
		      \begin{equation}
			      \taus \merge \tau.\trans{Q},
		      \end{equation}
		      which is rooted branching bisimilar via Lemma \cref{lma:tau-on-p-process}.
	\end{itemize}
	Therefore, our translation for Sliding Choice is valid up to rooted branching bisimulation.
\end{proof}

\subsection{Interrupt}\label{ssec:proof-interrupt}

From \cref{ssec:interrupt}, our translation of the \textbf{Interrupt} operator $\interrupt$ is defined as the following equation:
\begin{equation}\tag{\ref{trans:interrupt}}
	\trans{P \triangle Q} = \partial_{H_{0}}\Bigl(f_{\mathtt{post}}\Bigl[ f_{\mathtt{origin}}(\trans{P}) \merge \Pi \merge \Gamma(\trans{Q}) \Bigr]\Bigr)
\end{equation}

\begin{proof}
	We proceed in the same manner as the previous operators.

	\begin{itemize}
		\item Let $P'$ such that $P \prightarrow{a} P'$. Since every visible action in $P$ is tagged $a_{\mathtt{origin}}$, which can always communicate with $\mathtt{origin}\in H_{0}$, we can always apply Lemma \ref{lma:merge-simplification-h-h} (Note that $\Pi \xrightarrow{\mathtt{origin}} \Pi$)
		      \begin{align*}
			       & \partial_{H_{0}}\Bigl(f_{\mathtt{post}}\Bigl[ f_{\mathtt{origin}}(\trans{P}) \merge \Pi \merge \Gamma(\trans{Q}) \Bigr]\Bigr)\prightarrow{a} \\
			       & \partial_{H_{0}}\Bigl(f_{\mathtt{post}}\Bigl[ f_{\mathtt{origin}}(\trans{P'}) \merge \Pi \merge \Gamma(\trans{Q}) \Bigr]\Bigr)
		      \end{align*}
		      which is Strongly bisimilar to $P' \interrupt Q$.

		\item Let $P'$ such that $P \prightarrow{\tau} P'$. Since $\tau$ actions can never communicate, and we have that $\mathtt{split}, \mathtt{origin}\in H_{0}$, we can always apply Lemma \ref{lma:merge-simplification-a-h}:
		      \begin{align*}
			       & \partial_{H_{0}}\Bigl(f_{\mathtt{post}}\Bigl[ f_{\mathtt{origin}}(\trans{P}) \merge \Pi \merge \Gamma(\trans{Q}) \Bigr]\Bigr)\prightarrow{\tau} \\
			       & \partial_{H_{0}}\Bigl(f_{\mathtt{post}}\Bigl[ f_{\mathtt{origin}}(\trans{P'}) \merge \Pi \merge \Gamma(\trans{Q}) \Bigr]\Bigr)
		      \end{align*}
		      which is Strongly bisimilar to $P' \interrupt Q$.
		\item Let $Q'$ such that $Q \prightarrow{a} Q'$. At any step in $\Pi$, there is the possibility to take a $\mathtt{split}$ step, and communicate with $a_{\mathtt{ini}}$. Therefore, we can then apply Lemma \ref{lma:merge-simplification-h-h}:
		      \begin{align*}
			       & \partial_{H_{0}}\Bigl(f_{\mathtt{post}}\Bigl[ f_{\mathtt{origin}}(\trans{P}) \merge \Pi \merge \Gamma(\trans{Q}) \Bigr]\Bigr)\prightarrow{a} \\
			       & \partial_{H_{0}}\Bigl(f_{\mathtt{post}}\Bigl[ f_{\mathtt{origin}}(\trans{P}) \merge \trans{Q'} \Bigr]\Bigr)
		      \end{align*}
		      Via Lemma \ref{lma:stagnant-inis}, This process reduces to
		      \[\taus \merge \trans{Q'}\]
		      which is Branching bisimilar to $P \interrupt Q'$ via Lemma \ref{lma:tau-on-p-process}.

		\item Let $Q'$ such that $Q \prightarrow{\tau} Q'$. Since $\tau$ actions can never communicate, and we have that $\mathtt{split}, \mathtt{origin}\in H_{0}$, we can always apply Lemma \ref{lma:merge-simplification-a-h}:
		      \begin{align*}
			       & \partial_{H_{0}}\Bigl(f_{\mathtt{post}}\Bigl[ f_{\mathtt{origin}}(\trans{P}) \merge \Pi \merge \Gamma(\trans{Q}) \Bigr]\Bigr)\prightarrow{\tau} \\
			       & \partial_{H_{0}}\Bigl(f_{\mathtt{post}}\Bigl[ f_{\mathtt{origin}}(\trans{P}) \merge \Pi \merge \Gamma(\trans{Q'}) \Bigr]\Bigr)
		      \end{align*}
		      This process is Strongly bisimilar to $P \interrupt Q'$.
	\end{itemize}
	We have now exhausted all cases, and therefore we can conclude that our translation of $\csp$ Interrupt is valid up to rooted branching bisimulation.
\end{proof}

\subsection{Throw}\label{ssec:proof-throw}
From \cref{ssec:throw}, the \textbf{Throw} operator $\throw$ is defined as the following equation:
\begin{equation}\tag{\ref{trans:throw}}
	\trans{P \throw Q}     = \partial_{H_{0}}\Bigl(\fdef{post} \Bigl[ \fdef{split}(\trans{P}) | | \Pi.\trans{Q} \Bigr]\Bigr).
\end{equation}

\begin{proof}
	We proceed in the same manner as the previous operators.

	\begin{itemize}[leftmargin=*]
	\item Let $P'$ such that $P \prightarrow{\tau} P'$. Therefore, we have:
	      \[\fdef{split}(\trans{P}) = \tau.\fdef{split}(\trans{P'})\]
	      Since $\mathtt{split},\,\mathtt{origin}\in H_{0}$, and $\tau$ cannot communicate, we apply Lemma \ref{lma:merge-simplification-a-h}:
	      \begin{align*}
		      \trans{P \throw Q}     = & \partial_{H_{0}}\Bigl(\fdef{post} \Bigl[\fdef{split}(\trans{P}) | | \Pi.\trans{Q} \Bigr]\Bigr)                           \\
		      =                        & \partial_{H_{0}}\Bigl(\fdef{post} \Bigl[ \tau.\fdef{split}(\trans{P'}) | | \Pi.\trans{Q} \Bigr]\Bigr) \prightarrow{\tau} \\
		                               & \partial_{H_{0}}\Bigl(\fdef{post} \Bigl[ \fdef{split}(\trans{P'}) | | \Pi.\trans{Q} \Bigr]\Bigr)
	      \end{align*}
	      This process is strongly bisimilar to $P' \throw Q$.

	\item Let $P'$ such that $P \prightarrow{a} P'$, and $a\not\in A$ for the target set $A$. We will have that:
	      \[\fdef{split}(\trans{P}) = a_{\mathtt{origin}}.\fdef{split}(\trans{P'})\]
	      Since $\mathtt{split},\,\mathtt{origin}\in H_{0}$, and $a_{\mathtt{origin}} \mid \mathtt{origin} = a$, we apply Lemma \ref{lma:merge-simplification-h-h}:
	      \begin{align*}
		      \trans{P \throw Q}     = & \partial_{H_{0}}\Bigl(\fdef{post} \Bigl[ \fdef{split}(\trans{P}) | | \Pi.\trans{Q} \Bigr]\Bigr)                                      \\
		      =                        & \partial_{H_{0}}\Bigl(\fdef{post} \Bigl[ a_{\mathtt{origin}}.\fdef{split}(\trans{P'}) | | \Pi.\trans{Q} \Bigr]\Bigr) \prightarrow{a} \\
		                               & \partial_{H_{0}}\Bigl(\fdef{post} \Bigl[ \fdef{split}(\trans{P'}) | | \Pi.\trans{Q} \Bigr]\Bigr)
	      \end{align*}
	      This process is strongly bisimilar to $P' \throw Q$.

	\item Let $P'$ such that $P \prightarrow{a} P'$, and $a\in A$ for the target set $A$. We will have that:
	      \[\fdef{split}(\trans{P}) = a_{\mathtt{split}}.\fdef{split}(\trans{P'})\]
	      Since $\mathtt{split},\,\mathtt{origin}\in H_{0}$, and $a_{\mathtt{split}} \mid \mathtt{split} = a$, we apply Lemma \ref{lma:merge-simplification-h-h}:
	      \begin{align*}
		      \trans{P \throw Q}     = & \partial_{H_{0}}\Bigl(\fdef{post} \Bigl[ \fdef{split}(\trans{P}) | | \Pi.\trans{Q} \Bigr]\Bigr)                                     \\
		      =                        & \partial_{H_{0}}\Bigl(\fdef{post} \Bigl[ a_{\mathtt{split}}.\fdef{split}(\trans{P'}) | | \Pi.\trans{Q} \Bigr]\Bigr) \prightarrow{a} \\
		                               & \partial_{H_{0}}\Bigl(\fdef{post} \Bigl[ \fdef{split}(\trans{P'}) | | \trans{Q} \Bigr]\Bigr)
	      \end{align*}
	      Via Lemma \ref{lma:stagnant-inis}, this reduces down to the equation:
	      \[\tau^{*} | | \trans{Q}\]
	      This is rooted branching bisimilar to $Q$ via Lemma \ref{lma:tau-on-p-process}.
\end{itemize}
We have now exhausted all cases, and therefore we can conclude that our translation of $\csp$ Throw is valid up to rooted branching bisimulation.
	
\end{proof}


\subsection{Generalising}

From these proofs, we have shown that every translation of the $\csp$ operators are valid up to rooted branching bisimilarity.

% \begin{thm}[Maybe 2]{thm:tau-comm}{}
% 	The following diagram cannot be modelled in $\acp$ via Communication of two processes $a$ and $\tau$
% 	% https://q.uiver.app/#q=WzAsMyxbMCwwLCJcXGJ1bGxldCJdLFsyLDAsIlxcYnVsbGV0Il0sWzAsMiwiXFxidWxsZXQiXSxbMCwxLCJcXHRhdSJdLFswLDIsImEiLDJdXQ==
% 	\[\begin{tikzcd}[cramped]
% 			\bullet && \bullet \\
% 			\\
% 			\bullet
% 			\arrow["\tau", from=1-1, to=1-3]
% 			\arrow["a"', from=1-1, to=3-1]
% 		\end{tikzcd}\]
% \end{thm}

\end{document}
