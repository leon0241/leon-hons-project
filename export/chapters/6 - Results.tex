\documentclass[../hons_project.tex]{subfiles}
\begin{document}

\section{Results}
In this paper, we have demonstrated a translation from the language $\csp$ to the language $\acptf$, an extension of the language $\acpt$ to include a Functional Renaming operator. We have shown that this translation is valid up to rooted branching bisimulation, which is Compositional in the language $\acp$ and its related extensions. We can therefore say that $\acptf$ is \textbf{at least as expressive as $\csp$ up to rooted branching bisimulation}.

\section{Limitations and Future Work}
In our encoding, we are limited by the fact that internal actions will remain in communications, and hence hindering the possibility of a Strong Bisimulation. The behaviour of abstraction being a one-way operator, and $\tau$ actions being forgetful is an intentional and inherent part of $\acpt$ from an axiomatic level. This leads me to believe that rooted branching bisimilarity is the finest relation there is between variants of $\csp$ and $\acp$.
    
Further work could be done to study the relationship between these two languages, as there are many fragments and extensions to the language $\acp$ and there may be a different approach which avoids this conundrum, with a valid solution up to strong bisimilarity possibly existing for some fragment of $\acp$.
\end{document}
